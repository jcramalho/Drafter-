\chapter{Tipologias de atos normativos}
\label{tipologias}

A plataforma que se pretende criar deverá ter modelos pré-criados para todas as tipologias de atos normativos 
que se venham a suportar. 
Um modelo de uma tipologia pressupõe a especificação estrutural dos documentos pertencentes a essa tipologia, 
podendo também conter uma sugestão de texto a incluir nos vários elementos estruturais que a compoem.

No contexto legístico português há dezenas de tipologias. No âmbito deste trabalho e devido a 
restrições temporais, foi necessário reduzir a um subconjunto mas significativo e representativo do que se pretende.

Consultou-se a entidade adjudicante que designou como prioritárias a lei, o decreto-regulamentar, o decreto, 
a portaria e o despacho normativo 
(publicado na 2.a série do Diário da República) e, em certos casos, a resolução do Conselho de Ministros.

\section{Metadados}

Depois de uma análise feita sobre vários documentos de cada uma destas tipologias, chegou-se ao seguinte conjunto
de metadados, comuns a todas:

\begin{description}
    \item[referencia] - fórmula usualmente utilizada para identificar os diplomas; 
    Normalmente, os documentos dentro de uma tipologia 
    são referenciados por combinação de tipologia, número de série e ano;
    \item[tipologia] - designação da tipologia: \texttt{Lei}, \texttt{Decreto-lei}, etc;
    \item[localPublicacao] - local de publicação, 1.ª ou 2.ª Série do 
    Diário da República;
    \item[numPublicacao] - número de publicação, número atribuído sequencialmente 
    dentro do mesmo ano e da mesma tipologia;
    \item[dataPublicacao] - data de publicação do diploma;
    \item[emissor] - entidade emissora; Pode ser a Assembleia da República, o Governo, 
    a Presidência do Conselho de Ministros, um Ministério em Especial ou uma Secretaria 
    de Estado de algum Ministério;
    \item[sumario] - contém a indicação do assunto principal do diploma;
    \item[preambulo] - contém um enquadramento legal e justificativo do diploma, 
    que normalmente termina com a indicação de que a entidade emissora \emph{"decreta o seguinte"};
    \item[articulado] - contém a parte dispositiva do diploma, artigos/normas legais;
    \item[anexos] - pode conter ou não; são usualmente colocadas as tabelas, listagem, mapas, símbolos, 
    ou outros elementos gráficos ou quantitativos referidos no articulado.
\end{description}

No seguimento desta análise, referem-se exemplos reais de documentos nas tipologias selecionadas e apresentam-se 
exemplos de como seriam os respetivos registos de metadados em XML. Escolhe-se o XML por ser um formato aberto,
e por ser o formato a usar na plataforma a ser desenvolvida.

\section{Lei}

Um documento desta tipologia pode ser definido como um ato legislativo, emanado pela Assembleia da República, 
no exercício da sua função legislativa, ao abrigo do artigo 164.º e 165.º da Constituição da República Portuguesa.

As leis da Assembleia da República obedecem ao formulário seguinte:
\begin{quoting}
A Assembleia da República decreta, nos termos da alínea... do artigo 161.º da Constituição, o seguinte:
(Segue-se o texto)
\end{quoting}

Tratando-se de lei constitucional ou orgânica, deve mencionar-se expressamente o termo correspondente, 
na parte final da fórmula. 

Tratando-se de resoluções de aprovação de tratados ou acordos internacionais, o texto é composto do seguinte modo:
\begin{quoting}
Aprovar (para ratificação, no caso dos tratados) o ... 
(segue-se a identificação do tratado ou do acordo internacional em forma simplificada, 
com indicação da matéria a que respeita, do local e data da assinatura, sendo o teor do respetivo 
instrumento publicado em anexo).
\end{quoting}

\subsection{Exemplo: Lei n.º 74/98, de 11 de novembro} 

O seu registo de metadados teria a seguinte extrutura em XML:

\begin{Verbatim}[frame=single, numbers=left, fontsize=\small, commandchars=\\\{\}]
<?xml version="1.0" encoding="UTF-8"?>
<documento>
    <referencia>Lei 74/98</referencia>
    <tipologia>Lei</tipologia>
    <localPublicacao>1.ª Série</localPublicacao>
    <numPublicacao>74</numPublicacao>
    <dataPublicacao>1998-11-11</dataPublicacao>
    <emissor>Assembleia da República</emissor>
    <sumario>Publicação, identificação e formulário 
            dos diplomas.</sumario>
    <preambulo>A Assembleia da República decreta, 
            nos termos da alínea c) do artigo 161.º da 
            Constituição, para valer como lei geral da República, 
            o seguinte:</preambulo>
    <articulado>ver diploma</articulado>
    <anexos>ver diploma</anexos>
</documento>
\end{Verbatim}

Este documento XML, foi criado apenas para dar uma visão minimalista dos metadados num formato inteligível quer 
para a máquina quer para o humano. Na plataforma final, estes documentos terão de estar no formato \texttt{Akoma Ntoso}.
É um formato que traz mais alguma complexidade pois prevê a interoperabilidade internacional desta informação.
A título de exemplo, apresenta-se uma possível versão do documento acima neste formato.

\begin{Verbatim}[frame=single, numbers=left, fontsize=\footnotesize, commandchars=\\\{\}]
<?xml version="1.0" encoding="UTF-8"?>
  <akomaNtoso>
    <act>
      <meta>
        <identification>
          <FRBRWork>
            <FRBRthis value="/akn/pt/act/lei/1998-11-11/74"/>
            <FRBRuri value="/akn/pt/act/lei"/>
            <FRBRdate date="1998-11-11" name="Lei 74/98"/>
            <FRBRauthor href="#assembleia-da-republica" as="author"/>
            <FRBRcountry value="pt"/>
          </FRBRWork>
          <FRBRExpression>
            <FRBRthis value="/akn/pt/act/lei/1998-11-11/74@1998-11-11"/>
            <FRBRuri value="/akn/pt/act/lei/1998-11-11/74"/>
            <FRBRdate date="1998-11-11" name="expressed"/>
            <FRBRauthor href="#assembleia-da-republica" as="author"/>
            <FRBRlanguage language="pt"/>
          </FRBRExpression>
          <FRBRManifestation>
            <FRBRthis value="/akn/pt/act/lei/1998-11-11/74@1998-11-11"/>
            <FRBRuri value="/akn/pt/act/lei/1998-11-11/74@1998-11-11"/>
            <FRBRdate date="1998-11-11" name="manifested"/>
            <FRBRauthor href="#assembleia-da-republica" as="author"/>
          </FRBRManifestation>
        </identification>
        <publication>
          <published>
            <refersTo value="1ª Série"/>
            <number value="74"/>
            <date date="1998-11-11"/>
          </published>
        </publication>
        <references source="#assembleia-da-republica">
          <TLCOrganization id="assembleia-da-republica">
            <orgName>Assembleia da República</orgName>
          </TLCOrganization>
        </references>
      </meta>
      <body>
        <preamble>
          <p>A Assembleia da República decreta, nos termos da alínea c) 
            do artigo 161.º da Constituição, para valer como lei geral da 
            República, o seguinte:</p>
        </preamble>
        <article>
          <heading>Articulado</heading>
          <p>ver diploma</p>
        </article>
        <annex>
          <heading>Anexos</heading>
          <p>ver diploma</p>
        </annex>
      </body>
    </act>
  </akomaNtoso>        
\end{Verbatim}

Explicação da estrutura:
\begin{description}
\item[akomaNtoso]: Elemento raiz que contém todo o documento;
\item[act]: Elemento que representa o ato legislativo;
\item[meta]: Metadados do documento, incluindo identificação e publicação;
\item[identification]: Identificação do documento nas três fases (Work, Expression, Manifestation);
\item[FRBRWork]: Representa a obra em si, contendo a URI e data de promulgação;
\item[FRBRExpression]: Representa a expressão da obra, incluindo a data e idioma da publicação;
\item[FRBRManifestation]: Representa a manifestação física ou digital da expressão da obra;
\item[publication]: Detalhes da publicação, como série, número e data;
\item[preface]: Contém o preâmbulo do documento;
\item[preamble]: O texto do preâmbulo;
\item[body]: Corpo principal do documento, dividido em seções;
\item[section]: Seções do documento, como sumário, articulado e anexos;
\item[heading]: Título da seção;
\item[p]: Parágrafo de texto.
\end{description}

Este exemplo cobre a estrutura básica do documento original dentro do padrão Akoma Ntoso. 
Dependendo dos detalhes específicos e dos requisitos adicionais, como normas de citação ou referências cruzadas, 
a estrutura pode ser expandida.


\section{Decreto-lei}

Um documento desta tipologia pode ser definido como um ato legislativo, diploma do Governo, no exercício 
da sua função legislativa, ao abrigo do artigo 198.º da Constituição da República Portuguesa.

Os decretos-lei têm o seguinte formulário:

\begin{description}
    \item[Decretos-leis previstos na alínea a) do n.º 1 do artigo 198.º da Constituição]: 
    
    \begin{quoting}
        Nos termos da alínea a) do n.º 1 do artigo 198.º da Constituição, o Governo decreta o seguinte:
        (Segue-se o texto.)
    \end{quoting}

    \item[Decretos-leis previstos na alínea b) do n.º 1 do artigo 198.º da Constituição]: 
    
    \begin{quoting}
        No uso da autorização legislativa concedida pelo artigo... da Lei n.º ...., de... 
        de..., e nos termos da alínea b) do n.º 1 do artigo 198.º da Constituição, 
        o Governo decreta o seguinte:
        (Segue-se o texto.)
    \end{quoting}

    \item[Decretos-leis previstos na alínea c) do n.º 1 do artigo 198.º da Constituição]: 
    
    \begin{quoting}
        No desenvolvimento do regime jurídico estabelecido pela Lei (ou Decreto-Lei) n.º ...., 
        de... de..., e nos termos da alínea c) do n.º 1 do artigo 198.º da Constituição, 
        o Governo decreta o seguinte:
        (Segue-se o texto.)
    \end{quoting}

    \item[Decretos-leis previstos no n.º 2 do artigo 198.º da Constituição]: 
    
    \begin{quoting}
        Nos termos do disposto no n.º 2 do artigo 198.º da Constituição, o Governo decreta o seguinte:
        (Segue-se o texto.)
    \end{quoting}

\end{description}

\subsection{Exemplo: DL n.º 4/2024, de 5 de janeiro} 
    
O seu registo de metadados teria a seguinte extrutura em XML:
    
\begin{Verbatim}[frame=single, numbers=left, fontsize=\small, commandchars=\\\{\}]
<?xml version="1.0" encoding="UTF-8"?>
    <documento>
        <referencia>DL 4/2024</referencia>
        <tipologia>DL</tipologia>
        <localPublicacao>1.ª Série</localPublicacao>
        <numPublicacao>4</numPublicacao>
        <dataPublicacao>2024-01-05</dataPublicacao>
        <emissor>Presidência do Conselho de Ministros</emissor>
        <sumario>Institui o mercado voluntário de carbono e 
            estabelece as regras para o seu funcionamento.</sumario>
        <preambulo>ver diploma</preambulo>
        <articulado>ver diploma</articulado>
        <anexos>ver diploma</anexos>
    </documento>
\end{Verbatim}


\section{Decreto}

Um documento desta tipologia pode ser definido como um Diploma do Governo que visa aprovar 
os acordos internacionais, ao abrigo do artigo 197.º, n.º 1, al. c), da Constituição da 
República Portuguesa.

Os decretos têm o seguinte formulário:

\begin{quoting}
    Nos termos da alínea c) do n.º 1 do artigo 197.º da Constituição, o Governo aprova 
    o... (segue-se a identificação do acordo internacional em forma simplificada, com 
    indicação da matéria a que respeita, do local e da data da assinatura, sendo o 
    teor do respetivo instrumento publicado em anexo).
\end{quoting}


\subsection{Exemplo: Decreto 1/2024, de 22 de janeiro} 
    
O seu registo de metadados teria a seguinte extrutura em XML:
    
\begin{Verbatim}[frame=single, numbers=left, fontsize=\small, commandchars=\\\{\}]
<?xml version="1.0" encoding="UTF-8"?>
    <documento>
        <referencia>Decreto 1/2024</referencia>
        <tipologia>Decreto</tipologia>
        <localPublicacao>1.ª Série</localPublicacao>
        <numPublicacao>1</numPublicacao>
        <dataPublicacao>2024-01-22</dataPublicacao>
        <emissor>Presidência do Conselho de Ministros</emissor>
        <sumario>Aprova o Acordo de Cooperação Económica entre a 
            República Portuguesa e a República da Moldova.</sumario>
        <preambulo>ver diploma</preambulo>
        <articulado>ver diploma</articulado>
        <anexos>ver diploma</anexos>
    </documento>
\end{Verbatim}


\section{Decreto-Regulamentar}

Um documento desta tipologia pode ser definido como um Regulamento. É um diploma do Governo, 
no exercício da sua função administrativa, ao abrigo dos artigos 199.º, als. c) ou g), e 
112.º, n.º 6 . É pouco comum.

Os decretos-regulamentares têm o seguinte formulário:

\begin{quoting}
    Nos termos da alínea c) do artigo 199.º da Constituição e... 
    (segue-se a identificação do ato legislativo a regulamentar), 
    o Governo decreta o seguinte: : 
    (segue-se o texto)
\end{quoting}


\subsection{Exemplo: DR 3/2024, de 21 de fevereiro} 
    
O seu registo de metadados teria a seguinte extrutura em XML:
    
\begin{Verbatim}[frame=single, numbers=left, fontsize=\small, commandchars=\\\{\}]
<?xml version="1.0" encoding="UTF-8"?>
    <documento>
        <referencia>DR 3/2024</referencia>
        <tipologia>Decreto-Regulamentar</tipologia>
        <localPublicacao>1.ª Série</localPublicacao>
        <numPublicacao>3</numPublicacao>
        <dataPublicacao>2024-02-21</dataPublicacao>
        <emissor>Presidência do Conselho de Ministros</emissor>
        <sumario>Procede à fixação do universo dos contribuintes 
            abrangidos pela declaração automática de rendimentos.
        </sumario>
        <preambulo>ver diploma</preambulo>
        <articulado>ver diploma</articulado>
        <anexos>ver diploma</anexos>
    </documento>
\end{Verbatim}


\section{Resolução do Conselho de Ministros}

Um documento desta tipologia pode ser definido como uma resolução emanada quando 
o Governo reúne em plenário, concretiza-se o Conselho de Ministros. 
É um ato normativo do Governo no exercício da sua função administrativa.

As Resoluções do Conselho de Ministros têm o seguinte formulário:

\begin{quoting}
    Nos termos da alínea... do artigo 199.º da Constituição, o Conselho de Ministros 
    resolve:
        (Segue-se o texto.)
\end{quoting}

Ou:
\begin{quoting}
    Nos termos do... (segue-se a identificação do ato e da respetiva norma que 
    estabelece a exigência de resolução) e da alínea... do artigo 199.º da Constituição, 
    o Conselho de Ministros resolve:
    (Segue-se o texto.)
\end{quoting}


\subsection{Exemplo: RCM 28/2024, de 23 de fevereiro} 
    
O seu registo de metadados teria a seguinte extrutura em XML:
    
\begin{Verbatim}[frame=single, numbers=left, fontsize=\small, commandchars=\\\{\}]
<?xml version="1.0" encoding="UTF-8"?>
    <documento>
        <referencia>RCM 28/2024</referencia>
        <tipologia>RCM</tipologia>
        <localPublicacao>1.ª Série</localPublicacao>
        <numPublicacao>28</numPublicacao>
        <dataPublicacao>2024-02-23</dataPublicacao>
        <emissor>Presidência do Conselho de Ministros</emissor>
        <preambulo>ver diploma</preambulo>
        <articulado>ver diploma</articulado>
        <anexos>ver diploma</anexos>
    </documento>
\end{Verbatim}

O campo \texttt{sumário} considera-se não aplicável nesta tipologia.


\section{Portaria}

Um documento desta tipologia pode ser definido como um diploma do Governo, 
no exercício da sua função administrativa. 
É muito comum que a própria lei determine a sua execução mediante portaria.

As Portarias têm o seguinte formulário:

\begin{quoting}
    Manda o Governo, pelo... (indicar o membro ou membros competentes), 
    o seguinte: 
    (Segue texto)
\end{quoting}


\subsection{Exemplo: Portaria 68/2024, de 23 de Fevereiro} 
    
O seu registo de metadados teria a seguinte extrutura em XML:
    
\begin{Verbatim}[frame=single, numbers=left, fontsize=\small, commandchars=\\\{\}]
<?xml version="1.0" encoding="UTF-8"?>
    <documento>
        <referencia>Portaria 68/2024</referencia>
        <tipologia>Portaria</tipologia>
        <localPublicacao>1.ª Série</localPublicacao>
        <numPublicacao>68</numPublicacao>
        <dataPublicacao>2024-02-23</dataPublicacao>
        <emissor>Presidência do Conselho de Ministros</emissor>
        <sumario>Décima segunda alteração ao Regulamento Específico 
            do Domínio da Competitividade e Internacionalização
        </sumario>
        <preambulo>ver diploma</preambulo>
        <articulado>ver diploma</articulado>
        <anexos>ver diploma</anexos>
    </documento>
\end{Verbatim}


\section{Despacho Normativo}

Um documento desta tipologia pode ser definido como um diploma do Governo, 
no exercício da sua função administrativa. 

Os Despachos Normativos têm o seguinte formulário:

\begin{quoting}
    (Inicia por identificar o acto legislativo que lhe serve de base.)
\end{quoting}


\subsection{Exemplo: DN 1/2024, de 5 de janeiro} 
    
O seu registo de metadados teria a seguinte extrutura em XML:
    
\begin{Verbatim}[frame=single, numbers=left, fontsize=\small, commandchars=\\\{\}]
<?xml version="1.0" encoding="UTF-8"?>
    <documento>
        <referencia>DN 1/2024</referencia>
        <tipologia>DN</tipologia>
        <localPublicacao>2.ª Série</localPublicacao>
        <numPublicacao>1</numPublicacao>
        <dataPublicacao>2024-01-05</dataPublicacao>
        <emissor>Gabinete do Secretário de Estado do Turismo, 
            Comércio e Serviços (Ministério da Economia e Mar)
        </emissor>
        <sumario>Prorroga o prazo de apresentação de candidaturas ao 
            concurso específico da Linha Interior + Turismo, aberto 
            na sequência dos incêndios de 4 e 5 de agosto de 2023
        </sumario>
        <preambulo>ver diploma</preambulo>
        <articulado>ver diploma</articulado>
        <anexos>ver diploma</anexos>
    </documento>
\end{Verbatim}


\section{Akoma Ntoso XML}

Akoma Ntoso (Arquitetura para Gestão Orientada ao Conhecimento de Textos Normativos Africanos usando Padrões Abertos e Ontologias) 
é uma norma técnica internacional para representar documentos executivos, legislativos e judiciais de maneira estruturada, 
utilizando um vocabulário XML específico do domínio.

O termo \texttt{akoma ntoso} significa "corações ligados" na língua Akan da África Ocidental e, por essa razão, foi escolhido 
para designar esta norma XML. A sigla usada normalmente para designar este formato é \texttt{AKN}.

\subsection{Estrutura e composição}

A norma AKN fornece uma estrutura abrangente para representar documentos parlamentares, legislativos e judiciais 
num formato XML legível quer por máquinas quer por humanos. 

Tem os seguintes componentes:

\begin{description}
\item[Vocabulário XML]: Este define o mapeamento entre a estrutura dos documentos legislativos e suas representações equivalentes 
em XML. Essencialmente, traduz os elementos dos textos legislativos num formato estruturado que pode ser facilmente processado por 
máquinas;

\item[Esquema XML]: Especificação da estrutura e respetivas restrições dos documentos legislativos em XML. 
Oferece uma capacidade descritiva extensiva aos vários tipos de documentos legislativos, incluindo:
    \begin{itemize}
        \item Documentos Parlamentares: Projetos de lei, atos e registros parlamentares.
        \item Documentos Judiciários: Sentenças e opiniões judiciais.
        \item Publicações Governamentais: Diários oficiais e outros registros governamentais.
    \end{itemize}

\item[Nomenclatura]: A AKN recomenda uma nomenclatura para identificar unicamente as fontes legislativas, 
baseada no modelo Requisitos Funcionais para Registros Bibliográficos (FRBR). 
Isso garante que cada documento possa ser identificado e referenciado de forma única, facilitando a gestão e a recuperação.

\item[Definição de Tipo MIME]: Esta especifica o tipo MIME apropriado para documentos Akoma Ntoso, garantindo que sejam corretamente reconhecidos e processados por navegadores web e outros sistemas de software.

\end{description}

Ao aderir a esses padrões, o Akoma Ntoso possibilita a criação de documentos legais interoperáveis, reutilizáveis e acessíveis, facilitando a transparência e a eficiência nos processos legais e legislativos.


\section{Sumário}

Ao longo deste capítulo, foram caraterizadas as tipologias de atos normativos que serão 
consideradas neste trabalho conducente a uma prova de conceito.

Já existe um formato aberto definido para o intercâmbio deste tipo de informação a nível europeu 
(Akoma Ntoso XML\footnote{\url{https://www.oasis-open.org/standard/akn-v1-0/}}, 
uma norma OASIS\footnote{\url{https://www.oasis-open.org/org/}} para documentos legislativos).
Esse formato está especificado num XML Schema, por isso se adiantou com uma possível
instância XML para cada um dos tipos de registo.

