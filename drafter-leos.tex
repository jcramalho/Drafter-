\chapter{LEOS: Legislation Editing Open Software}

\section{Introdução}

O LEOS é um projeto no âmbito da iniciativa \emph{"Interoperable Europe"} da Comissão Europeia para uma política reforçada de 
interoperabilidade do setor público, financiado pelo Programa Digital \emph{Europe (DIGITAL)} e criado para atender à 
necessidade da administração pública e das Instituições Europeias de gerar projetos de legislação em formato XML jurídico.

O projeto LEOS concentra-se em apoiar o co-desenvolvimento, co-design e co-implementação de um "ecossistema de Tecnologias de 
Informação (TI) centrado num LEOS aumentado".

O LEOS foi criado para abordar a modernização e transformação digital da elaboração e revisão de legislação nas 
Instituições da UE, agências e órgãos da UE e Estados-Membros.

Esta plataforma garante que o conteúdo elaborado pelos utilizadores siga as diretrizes de redação, oferecendo recursos como a
aplicação de estruturas de documento pré-definidas, layout pré-definido e regras de numeração. 
Tudo isso para garantir que o autor se possa focar na elaboração do texto e muito menos na gestão do layout (ou verificação). 
Para facilitar a colaboração online eficiente, o LEOS também possui outros recursos como comentários, sugestões, 
controle de versão, edição colaborativa, etc.

