\chapter{Protótipo/Prova de Conceito}
\label{prototipo}
\section{Introdução}

Além dos relatórios produzidos, será desenvolvida uma prova de conceito, a uma escala
reduzida, que permitirá elucidar alguns dos requisitos e, provavelmente, levantar novos requisitos ainda
não especificados.

A prova de conceito a desenvolver será composta pelas seguintes atividades e respetivos
resultados:

\begin{itemize}
\item Adoção do software open source LEOS (Legislation Editing Open Software ), como base da
solução, instalação e disponibilização online;
\item Colheita de um subconjunto de legislação do DRE;
\item Colheita de algumas bases de dados de jurisprudência dos tribunais;
\item Especificação de um modelo ontológico para a legislação colhida (baseada no trabalho já
realizado pelo EPO no ELI);
\item Processamento/Mineração, usando técnicas de NLP (Natural Language Processing ), da
legislação colhida para extração de dados para o povoamento da ontologia especificada;
\item Disponibilização da ontologia através de um motor de gestão de bases de dados orientadas a
grafos online;
\item Disponibilização de uma interface de pesquisa baseada em SPARQL que permitirá navegar na
ontologia;
\item Integração do LEOS com a base de dados ontológica: como suporte à edição de legislação;
\item  (Possibilidade) Identificar os vários tipos de documentos legislativos e estudar a hipótese de
aplicar técnicas de Machine Learning (ML) para gerar automaticamente conteúdo novo no
documento que está a ser editado.
\end{itemize}

\section{LEOS: Legislation Editing Open Software}

O LEOS é um projeto no âmbito da iniciativa \emph{"Interoperable Europe"} da Comissão Europeia para uma política reforçada de 
interoperabilidade do setor público, financiado pelo Programa Digital \emph{Europe (DIGITAL)} e criado para atender à 
necessidade da administração pública e das Instituições Europeias de gerar projetos de legislação em formato XML jurídico.

O projeto LEOS concentra-se em apoiar o co-desenvolvimento, co-design e co-implementação de um "ecossistema de Tecnologias de 
Informação (TI) centrado num LEOS aumentado".

O LEOS foi criado para abordar a modernização e transformação digital da elaboração e revisão de legislação nas 
Instituições da UE, agências e órgãos da UE e Estados-Membros.

Esta plataforma garante que o conteúdo elaborado pelos utilizadores siga as diretrizes de redação, oferecendo recursos como a
aplicação de estruturas de documento pré-definidas, layout pré-definido e regras de numeração. 
Tudo isso para garantir que o autor se possa focar na elaboração do texto e muito menos na gestão do layout (ou verificação). 
Para facilitar a colaboração online eficiente, o LEOS também possui outros recursos como comentários, sugestões, 
controle de versão, edição colaborativa, etc.

