\chapter{Entidade Executante}

O Departamento de Informática da Universidade do Minho (DIUM) tem por missão a divulgação do conhecimento, 
fundamental e especializado, nas áreas da ciência e das tecnologias da computação, com particular destaque para a 
Programação associada à Verificação e Segurança, os Sistemas Inteligentes, os Sistemas Distribuídos e confiáveis, 
os Sistemas de Computação de Alto-desempenho, a Engenharia de Software e as Comunicações e Redes de Computadores.

Aposta numa abordagem rigorosa à resolução de problemas por computador com base na adopção de modelos formais e 
métodos sistemáticos de análise e desenvolvimento. Cumpre a sua missão:

\begin{itemize}
    \item Lecionando cursos de licenciatura, e pós-graduação: mestrado e doutoramento;
    \item Realizando projetos de investigação e desenvolvimento internos e externos à Universidade.
\end{itemize}

Conta para isso com um pessoal permanente de cerca de 52 Docentes (todos doutorados) e 10 técnicos e mais de uma dezena 
de professores convidados para reforço das várias equipes docentes. Aos cursos que oferece, assegura um nível de ensino 
de qualidade elevada, demonstrada quer pelo avultado número de candidatos às suas ofertas formativas, quer pela grande e 
continuada procura dos estudantes formados pelo DIUM por parte dos empregadores nacionais e estrangeiros.

Para criar e manter actual o conhecimento que ensina e aplica, a actividade de investigação dos seus docentes está enquadrada 
em vários centros de investigação. Aqui exploram a teoria e desenvolvem projetos de concretização, com a colaboração de bolseiros 
de vários níveis (desde iniciação à investigação a pós-doutorados), Associação de Estudantesde pós-graduação e de pós-doutoramento.


\section{Informação de Contacto}

\begin{center}
\begin{tabular}{l|r}
Endereço Web                 & \url{http://www.di.uminho.pt} \\\hline
Telefone                            & +351 253 604430 \\\hline
Correio electrónico          & jcr@di.uminho.pt \\\hline
Responsável do projeto       & José Carlos Ramalho \\\hline
Morada                       & Departamento de Informática\\
                             & Universidade do Minho\\
                             & 4710-057 Gualtar, Braga\\\hline                         
\end{tabular}
\end{center}