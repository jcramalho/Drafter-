\chapter{Requisitos}
\label{requisitos}

Na proposta subjacente ao contrato ao abrigo do qual se redige este relatório, constavam vários grupos de requisitos.
Neste capítulo, contextualiza-se cada um deles de acordo com o protótipo (cap. \ref{prototipo}) instalado, parametrizado 
e desenvolvido.

Relativamente ao protótipo, era um requisito que este se baseasse na plataforma LEOS (\emph{"Legislation Editing Open Software"}).
O LEOS tem já desenvolvidos e implementados muitos dos requisitos, carecendo de uma adaptação à realidade portuguesa.
Ao longo deste capítulo, faz-se uma associação de cada um dos requisitos a uma ou mais funcionalidades da plataforma LEOS,
indicando as parametrizações que será necessário fazer. Para os casos em que a associação não seja possível indicam-se as 
funcionalidades a desenvolver.

A criação desta nova plataforma pressupõe também uma ligação automática (interoperabilidade) a outro sistema onde estaria 
a jurisprudência nacional. Para isso, seria necessário que existisse uma normalização sobre a jurisprudência e que esta fosse 
disponibilizada numa API de dados. Algo que poderá acontecer num futuro próximo.

\section{Requisitos}

Nas subsecções seguintes descrevem-se os conceitos associados a cada requisito, se estão ou não presentes no LEOS e,
o que será preciso fazer caos não estejam.

\subsection{Requisitos funcionais}

Alguns destes requisitos estão relacionados com políticas sobre a adoção de normas e a especificação e 
criação de processos, no entanto,
ao colocarmos o LEOS como base, algumas destas decisões já foram tomadas e parte deste trabalho já está feita.

\subsubsection{Deteção automática de cumprimento de normas de produção de atos normativos}

O LEOS precisa de ser parametrizado com os modelos estruturais dos atos normativos que se pretendem criar.
No capítulo anterior, apresentou-se uma análise prévia necessária à criação destes modelos. 
O passo seguinte, será 
especificar esses modelos em XML numa linguagem específica desenvolvida para o efeito \cite{VPCB2018}.
Apresentam-se exemplos no capítulo referente à configuração da plataforma \ref{modelos}.

Esta especificação de cada um dos modelos pode e deve incluir os requisitos estruturais que devem ser observados na 
produção de atos normativos. Ao fazê-lo, o LEOS ou outra plataforma que os venha a utilizar pode verificar automaticamente 
o cumprimento dos requisitos/normas.

A linguagem definida pela comunidade para ser usada neste tipo de plataformas, o AKN (\ref{akn}), permite 
ir além dos requisitos estruturais podendo ser especificados requisitos de semântica dinâmica (dependentes 
do conteúdo que se vai introduzindo no documento).

\subsubsection{Observação de interpretações firmadas em jurisprudência sobre normas}

Este requisito pressupõe a verificação de algumas condições estruturais e semânticas.
As condições estruturais podem e devem estar refletidas no modelo criado em AKN (\ref{akn}) para a tipologia.
As condições semânticas podem ser divididas em duas: estáticas e dinâmicas.
Ambas terão de ser especificadas no momento da definição do modelo para a tipologia. 
A semântica estática pode ficar definida no modelo em AKN ou ainda numa estensão a este modelo criada na plataforma LEOS.
A semântica dinâmica, depende de valores que serão introduzidos aquando da criação do documento e a sua verificação terá 
de ser garantida usando uma parte da linguagem AKN, ou uma linguagem extra de regras ou materializadas 
no código da aplicação. Estas três alternativas dão solução ao problema, caberá à equipa de projeto 
selecionar a metodologia que for mais conveniente.

\subsubsection{Apoio à elaboração de tarefas de avaliação normativa}

As tarefas de avaliação normativa pretendem avaliar o impacto da nova legislação que está a ser criada.
É algo externo ao ato normativo e para o qual é necessário a interpretação e análise humana.
O apoio referido será no registo da informação que possa conduzir a um valor final de impacto. 

O LEOS não possui qualquer suporte à elaboração destas tarefas.
Para isso, será necessário desenvolver um componente de raiz, que poderá estar integrado no LEOS ou comunicar com este 
sendo um serviço externo.

O desenvolvimento deste componente carece de uma especificação cuidada de requisitos.

\subsubsection{Automatização dos processos de avaliação legislativa para textos preparados na plataforma}

Relacinonado com o ponto anterior. Estas funcionalidades deverão fazer parte do novo componente a desenvolver.

\subsubsection{Verificação e validação das referências normativas e legais identificadas nos textos preparados
na plataforma (verificação da existência das normas invocadas)}
\label{bases_legis}

Como já foi referido em cima, para o cumprimento deste requisito é necessário dispôr de uma base de dados com a 
produção jurídico-normativa de Portugal.

Para isso, é necessário garantir a adoção de várias regras/normas:

\begin{itemize}
    \item É preciso garantir uma forma única de referenciar os atos normativos, seja ela o ELI 
    ("\emph{European Legislation Identifier}") \cite{ELI}, ou simplesmente aquilo a que estamos habituados em Portugal, ou 
    seja, uma referência composta por tipologia, número de série e ano (no caso desta última, deverá 
    estar formalmente definida sem margem para ambiguidades); 

    \item Garantir a presença \emph{online}, de um repositório de atos normativos permanentemente atualizado e acessível;
    \item Ter uma API ("\emph{Application Program Interface}") bem definida sobre o repositório anterior que permita a 
    integração automática e a interoperabilidade com outros sistemas, por exemplo, o que se pretende desenvolver.
\end{itemize}

Devido à inexistência deste contexto de normalização, podemos avançar com um pequeno protótipo próprio 
mas que terá sempre as desvantagens de 
estar fora dos canais oficiais e de sofrer de permanente desatualização.


\subsubsection{Pesquisa e identificação automática de legislação e jurisprudência}
\label{bases_jurisprud}

Os requisitos descritos no ponto anterior são também válidos neste ponto.
Havendo um repositório de legislação que obedeça a um conjunto de regras transversais, basta incluir uma funcionalidade de 
pesquisa na sua API.

Relativamente à jurisprudência, a situação é semelhante à da legislação, é necessário garantir a adoção de várias regras/normas:

\begin{itemize}
    \item É preciso garantir uma forma única de referenciar os processos. Para isto já existe o ECLI 
    ("\emph{European Case Law Identifier}") \cite{ECLI}. É preciso promover/decretar a sua adoção à escala nacional; 

    \item Garantir uma normalização a nível dos metadados à escala nacional, todas as instituições produtoras de 
   jurisprudência deverão produzi-la na mesma forma, o que não se passa atualmente;

    \item Garantir a presença \emph{online}, de um repositório de jurisprudência permanentemente atualizado e acessível;

    \item Ter uma API ("\emph{Application Program Interface}") bem definida sobre o repositório anterior que permita a 
    integração automática e a interoperabilidade com outros sistemas, por exemplo, o que se pretende desenvolver.
\end{itemize}

À semelhança do ponto anterior, devido à inexistência deste contexto normalizado, podemos avançar 
com um pequeno protótipo próprio mas que terá sempre as desvantagens de 
estar fora dos canais oficiais e de sofrer de permanente desatualização.

\subsubsection{Verificação semântica das normas invocadas}

Para suportar este requisito será preciso ir mais além da criação dos modelos AKN para as tipologias.
Será necessário criar um modelo semântico para cada uma, aquilo que se designa por ontologia.

Neste contexto, uma ontologia pode definir-se como uma especificação formal de conhecimento de um determinado domínio bem 
caraterizado e que inteligível quer para máquinas quer para humanos.

Há várias linguagens que permitem a criação de ontologias com níveis diferentes de aplicabilidade.
Aquela que melhor se adapta à complexidade do contexto jurídico-normativo é a "\emph{Ontology Web Language} (OWL)" 
\cite{owl,owl2}.

A OWL é uma linguagem de anotação desenhada para especificar semântica, publicar e partilhar dados. 
Foi desenvolvida pelo \emph{World Wide Web Consortium} (W3C).

Um ontologia OWL é um conjunto de conceitos, seus atributos e relações entre eles. 
Adicionalmente, permite a especificação de regras semânticas via axiomas.

As ontologias são uma maneira de modelar o conhecimento de uma forma estruturada, permitindo que diferentes sistemas 
compreendam e usem esses dados de forma interoperável.

A linguagem OWL é baseada em \emph{Description Logic} (DL), que fornece uma base formal para o raciocínio automatizado. 
Isso significa que a partir de uma ontologia de base a máquina consegue inferir novos fatos a partir dos dados existentes, 
aumentando automaticamente o seu conhecimento e com isso 
melhorar a busca de informação e facilitar a integração de dados de diversas fontes.
A linguagem tem vários níveis de complexidade deixando a quem a utiliza a decisão de qual o nível que pretende usar. 
Desta forma, pode-se dizer que tudo se pode especificar em OWL.

Esta tarefa carece de uma especificação formal dos modelos semânticos que deve ser feita em conjunto com a 
especificação formal dos modelos das tipologias.
O modelo semântico OWL poderá até, inicialmente, ser povoado com a informação proveniente da especificação 
da tipologia.

\subsubsection{Descrição dos workflows para a criação e gestão de atos normativos}

O Ministério da Justiça possui já uma ferramenta própria para automatização de workflows. 
Se for possível colocar o novo sistema a dialogar com essa ferramenta consegue-se cumprir este requisito e 
poupar imenso tempo de desenvolvimento.

Para isso, será necessário garantir a interoperabilidade técnica e sintática entre os dois sistemas. 
Ou seja, terá de haver mecanismos/protocolos para o intercâmbio de informação entre os dois sistemas e 
deverá estar definido o formato da informação que se vai trocar.


\subsection{Requisitos de interoperabilidade}

Hoje em dia, falar de interoperabilidade é quase um requisito mas muitos desconhecem que esta é complexa 
e pode ser considerada em vários e diferentes níveis.

Da bibliografia, interoperabilidade é a capacidade de um sistema (informatizado ou não) de comunicar de 
forma transparente (ou o mais próximo disso) com outro sistema (semelhante ou não).

A nossa Agência para a Modernização Administrativa (AMA)  definiu um modelo para a interoperabilidade na 
Administração Pública (AP):
\begin{itemize}
\item Neste contexto, a interoperabilidade consiste na capacidade das organizações interagirem 
e agirem em prol de benefícios comuns, através de comunicação e partilha de informação e conhecimento;
\item O modelo foi baseado na \emph{Framework Europeia de Interoperabilidade} \cite{EIF}, criada pela Comissão Europeia;
\item Está organizado em quatro camadas: legal, organizacional, técnica e semântica.
\end{itemize}

Apesar de independentes, as quatro camadas são interdependentes, sendo que para se conseguir atingir a interoperabilidade 
semântica, as outras terão de estar contempladas (neste projeto é importante atingir o patamar semântico).

No contexto da interoperabilidade, pretende-se uma integração com as fontes primárias fundamentais, 
designadamente bases de dados, com toda a
legislação e atos normativos, para apoio à redação legislativa e normativa: p.e., Diário da
República. Como já foi referido atrás (\ref{bases_legis}), este é um requisito com implicações
organizacionais e políticas, é preciso normalizar e adoptar transversalmente algumas regras. Antes da 
interoperabilidate técnica, sintática e semântica, temos de resolver os requisitos da 
interoperabilidade organizacional.

Pretende-se também a integração com fontes primárias de jurisprudência, que é também um problema 
que começa no topo com a interoperabilidade organizacional (\ref{bases_jurisprud}). É preciso normalizar 
e adotar transversalmente um conjunto de regras já expostas anteriormente.

A interoperabilidade sintática e semântica fica garantida com a adoção do Akoma Ntoso \cite{AkomaNtoso2018} 
quer para a definição estrutural e semênticas das tipologias legísticas como para o seu conteúdo.

Por fim, para garantir a interoperabilidade técnica será preciso expor uma API de dados REST ou Web
Service a partir do LEOS, o que com algum trabalho se consegue fazer.


\subsection{Identificação dos mecanismos de IA e das ferramentas conexas a usar no desenvolvimento
da plataforma}

Relativamente a este assunto, a equipa do LEOS, esta equipa faz parte do grupo técnico que está a 
desenvolver soluções de interoperabilidade na Comunidade Europeia, realizou algum trabalho muito 
importante e que permite avançar muito na compreensão e do que importa desenvolver nesta área.

Durante a segunda metade de 2023, reuniram um grupo de trabalho composto por advogados, 
políticos ligados à criação de novas políticas e linguístas, com o objetivo de identificar 
as funcionalidades inteligentes que eles considerassem mais úteis no contexto em causa.
O seu contributo permitiu a identificação das seguintes funcionalidades:

\begin{itemize}
    \item Correlação entre considerandos e os termos do ato normativo em construção;
    \item Identificar automaticamente a legislação existente relevante para o ato em desenvolvimento;
    \item Identificar siglas, organizações e outras abreviaturas;
    \item Usar formulações linguísticas corretas dentro da estrutura do documento;
    \item Formulação correta de acordo com o Guia de Estilo em Português (ou outro que venha a ser produzido);
    \item Detectar divergências entre diferentes traduções linguísticas;
    \item Sugerir formulações linguísticas em disposições;
    \item Detectar e evitar formulações que possam criar problemas na interpretação jurídica;
    \item Correlação entre atos anteriores e o novo que se está a criar;
    \item Detectar obrigações, direitos, autorizações e  penalidades;
    \item Geração de texto jurídico com base em LLM (\emph{"Large Language Model"}: modelo de linguagem de grande escala).
\end{itemize}

Ainda antes deste trabalho, com início em 2021, foi elaborado um estudo \cite{PFPD22}, 
pela universidade de Bolonha que explorou a aplicação de inteligência artificial neste contexto.

Este estudo baseou-se na seguinte premissa:

\begin{quoting}
    Um ecossistema de tecnológico bem integrado, com um 'LEOS Aumentado' no seu núcleo, 
    tem o potencial de transformar digitalmente os processos legislativos, com um impacto 
    significativo na qualidade, eficiência e transparência.
\end{quoting}

Os seus objetivos principais foram explorar as oportunidades oferecidas pela IA híbrida para aprimorar 
a redação legal e melhorar a qualidade, eficiência e transparência da elaboração de leis e discutir 
ações para aproveitar a mudança digital na redação de legislação usando IA de maneira justa, 
responsável, ética e "explicável". 

Pretendia-se melhorar a qualidade do conteúdo legal e do processo de elaboração de leis 
investigando várias vertentes do problema, entre as quais as seguintes:

\begin{itemize}

\item clareza textual que apoia os redatores legais e a apresentação ao utilizador final, 
incluindo acessibilidade e visualização (design legal);
\item variantes linguísticas e gestão de versões temporais de cada tipo de documento legislativo;
\item elaboração de leis e políticas através de todas as etapas no processo decisório da Comissão, 
apoiando, por exemplo, emendas e a consolidação de emendas;
\item consistência de metadados (ELI, ECLI, AKN, CDM, etc.) através de todas as diferentes etapas 
de elaboração de leis para garantir validade legal ao longo do tempo e para preservação a longo prazo;
\item raciocínio lógico usando normas legais expressas no documento legislativo; 
\item facilitação dos Estados-Membros na implementação da lei, possibilitando o acompanhamento da 
transposição da lei e apoiando sua adoção.

\end{itemize}

Para melhorar a eficiência o estudo focou-se nas seguintes características:

\begin{itemize}

\item reduzir o trabalho manual/sujeito a erros utilizando padrões (por exemplo, erratas) e modelos 
de melhores práticas no processo de redação legal para automatizar o máximo possível a consolidação 
e a anotação semântica, usando ontologias legais e tesauros (por exemplo, EuroVoc);
\item maximizar a reutilização de conceitos legais similares detectados usando aprendizagem automática 
e análise de dados legais aplicados a todo o sistema jurídico (por exemplo, definição, 
análise de derrogações, exceções);
\item auxiliar a implementação de prioridades políticas na legislação (por exemplo, prontidão digital, 
neutralidade de gênero); 
\item melhorar a transparência e a capacidade de busca até à publicação.

\end{itemize}

Finalmente, o estudo propõe cenários inovadores e uma visão abrangente para, nos próximos cinco anos, 
fazer um progresso significativo em direção a um processo decisório totalmente digital, 
adotando tecnologias inovadoras enquanto preserva o rigor legal.

\subsubsection{Constrangimentos e Requisitos no uso de IA}

Para se avançar neste contexto tecnológico, foram identificados alguns constrangimentos 
que é preciso verificar e alguns requisitos que é preciso garantir. 
Quer uns quer outros são de transposição imediata para o nosso contexto em Portugal.

Os principais constrangimentos identificados são:

\begin{itemize}
    \item Culturais e institucionais: vontade de assumir riscos e embarcar na 
    mudança; é preciso criar e aumentar o conhecimento no domínio tecnológico 
    aplicado à redação da legislação;
    \item Técnicos: verificar se a maturidade tecnológica é suficiente; 
    poderá haver um distanciamento entre os casos de uso e a tecnologia existente;
    \item Recursos: há orçamento suficiente? há pessoal técnico com as 
    competências necessárias? qual a capacidade de aprendizagem da organização? 
    \item Legais: a utilização da IA na redação de atos normativos deve respeitar a
    recente legislação europeia aprovada sobre o assunto; é preciso assegurar 
    a transparência e eliminar qualquer parcialidade.
\end{itemize}

E os requisitos que é necessário assegurar são:
\begin{itemize}
    \item Envolvimento por parte dos orgãos de decisão: disponibilização dos 
    recursos necessários à implementação do projeto;
    \item Envolvimento transversal com o sistema e as práticas que se estão a desenvolver;
    \item Criar as parcerias institucionais necessárias a uma adoção transversal.
\end{itemize}


\subsubsection{Akoma Ntoso: a Inteligência Artificial precisa de uma base de representação sólida}

Nos últimos vinte anos, testemunhamos uma evolução na digitalização de fontes 
jurídicas, especialmente as legislativas. Este processo foi sofrendo uma evolução
natural.

De início, digitalizavam-se os jornais oficiais, no nosso caso, o Diário da República, para fornecer acesso aberto à 
legislação online usando a Web. 

Em seguida, com o emergir da "Web de Dados", transformaram-se as informações incluídas na legislação em dados abertos usando 
normas abertas (noutros países, em Portugal não existe esta prática). 

Pouco depois, aplicou-se o mesmo padrão para uma transformação digital profunda do processo de elaboração de leis, 
ao mesmo tempo em que se aprimoraram os fluxos de trabalho entre instituições. 

O passo seguinte é usar todos os dados abertos e a estrutura documental representada em padrões abertos para permitir a 
análise de dados jurídicos, de modo a criar novas aplicações de IA utilizando esses grandes volumes de dados jurídicos e 
também transformar partes das regras processuais em contratos inteligentes que são imediatamente executáveis e aplicáveis.

Vários jornais oficiais, arquivos nacionais e parlamentos tentaram gerir fontes jurídicas dentro de corpora jurídicos com o uso 
de tecnologias como bases de dados, XML, metadados RDF e fórmulas lógicas. 
Subsequentemente, eles também começaram a fornecer versões atualizadas da legislação a qualquer momento 
(o chamado mecanismo "point-in-time"). 
Em 1995, o EnAct (Arnold-Moore 1995), pelo Governo da Tasmânia, foi o primeiro sistema a produzir uma base de dados legislativa 
"point-in-time" em SGML. 
Em 1992, o LII (Legal Information Institute) da Cornell Law School, lançado por Peter Martin e Tom Bruce (Bruce 1994), 
disponibilizou na web (HTML) o Código Consolidado dos Estados Unidos. 

O AustLII, o Instituto de Informação Jurídica da Australásia, cofundado por Graham Greenleaf em 1995, disponibiliza instrumentos 
de IA como o DataLex. Esses instrumentos são baseados em software de inferência jurídica baseado em regras, capaz de dialogar 
com o utilizador final (Greenleaf 2020).

Usando Formex, uma linguagem SGML agora traduzida para XML (Formex v4), o Eur-Lex começou a consolidar a base de dados da 
legislação europeia em 1999. 
Em 1 de janeiro de 2001, a Noruega ativou um serviço web pela Lovdata e começou a fornecer legislação consolidada. 

Em 2002, a França transformou o serviço comercial Jurifrance num portal web público chamado Legifrance. 
O Legifrance inclui textos consolidados em formato misto (HTML, XML, PDF). Hoje, também suporta o formato Akoma Ntoso.

A Áustria lançou o projeto eLaw (2004) e transformou a sua base de dados RIS anterior (1983) numa coleção web de documentos 
autênticos, desmaterializando completamente a publicação do seu jornal oficial. 

A Região da Emília-Romanha (Itália) começou a consolidar regulamentos em 2003 usando o esquema XML NormeInRete. 
O Tribunal de Cassação da Itália começou com uma iniciativa semelhante em 2005 e agora está 
quase a atingir a consolidação de todo o corpora de documentos. 
O Senado da Itália adotou o padrão Akoma Ntoso para projetos de lei, transcrições e outros tipos de documentos também 
fornecidos em Open Government Data.

Em 30 de junho de 2009, o Senado do Brasil lançou o banco de dados parlamentar consolidado (LexMLBrasil) com uma função 
"point-in-time" baseada numa parametrização do esquema XML Akoma Ntoso. 

A Biblioteca do Congresso do Chile também fornece legislação atualizada usando Akoma Ntoso. 
Os Arquivos Nacionais do Reino Unido vêm progressivamente transformando toda a legislação do Reino Unido em XML, RDF e 
Akoma Ntoso desde 2012. 

O Kenya Law Report está agora a converter a sua base de dados de leis em documentos XML marcados de acordo com o Akoma Ntoso. 
Em 2017, as Nações Unidas aprovaram o Akoma Ntoso como norma oficial para sua documentação (AKN4UN), e as instituições da UE 
lançaram um projeto similar em 2018 com a iniciativa AKN4EU.

A interoperabilidade entre instituições e a simplificação do processo de elaboração de leis entre diferentes órgãos é facilitada 
pelo uso de padrões jurídicos XML. 
Além disso, a modelagem LegalXML fornece uma serialização digital robusta e sólida de documentos jurídicos ao aplicar 
princípios da teoria jurídica para anotar conhecimento jurídico (por exemplo, parâmetros temporais, anotação de web semântica, 
estrutura documental, citações normativas).

\subsubsection{Modelo híbrido de tecnologias de IA}

Muitas iniciativas estão em curso, em vários países, no contexto da aplicação de ferramentas de IA à produção de atos 
normativos. Convem no entanto realçar, que a solução não virá de uma tecnologia em específico mas de uma combinação de 
tecnologias devido a uma série de requisitos específicos deste domínio:

\begin{itemize}
    \item As tecnologias de "Machine Learning/Deep Learning (ML/DL)" funcionam sem lógica ou semântica, e muita informação 
    contextual incluída no documento jurídico é negligenciada, com uma evidente menor capacidade de interpretação;

    \item As citações jurídicas são uma prática consolidada nas disciplinas jurídicas, que confiam um papel importante a 
    recursos textuais externos (por exemplo, definições, derrogações, modificações, integração de prescritividade, penalidades, 
    condições). Isto significa que os algoritmos de ML/DL também devem considerar o texto citado, especialmente considerando 
    que alguns algoritmos (por exemplo, similitude, agrupamento) podem encontrar similaridades em textos 
    (por exemplo, "art. 3" e "art. 13") quando o conteúdo é completamente diferente. Por essa razão, a rede de normas através 
    de citações deve ser incluída na base de treino deste algoritmos;

    \item Os parâmetros temporais são fundamentais para criar um conjunto de dados robusto para ML/DL. 
    Por exemplo, jurisprudências baseadas em legislação revogada devem ter menor relevância no conjunto de dados que alimenta 
    o tratamento de IA de jurisprudências baseadas em nova legislação, mesmo que esses conjuntos de dados sejam menos frequentes. 
    Portanto, frequência, cálculo probabilístico e séries temporais devem ser mitigados com critérios de relevância e validade 
    jurídica;

    \item A anotação lógica e de web semântica também deve ser integrada nos algoritmos de ML/DL para ser possível 
    entender o tipo e o significado das relações que ligam diferentes sentenças no texto (por exemplo, obrigação e 
    penalidade, obrigação e derrogação).

\end{itemize}

Por estas razões, uma arquitetura híbrida, envolvendo anotações lógicas e semânticas, tratamento de citações e de referências 
temporais, será necessária. O AKoma Ntoso pode funcionar como o denominador comum no meio destas tecnologias.


\subsubsection{IA: implicações éticas e legais}

A aplicação de IA no domínio da redação de atos normativos tem diferentes implicações. 
Do ponto de vista constitucional, levanta a questão da delicada relação entre a separação democrática dos poderes; 
do ponto de vista da teoria jurídica, levanta o problema da interpretação que está oculta no código computacional, 
de modo que os princípios de texto aberto não são respeitados; da perspectiva ética, surgem questões quando a máquina 
indica uma regulamentação que pode levar a uma possível discriminação ou limitar a autonomia na tomada de decisões políticas.

Numa adoção da IA neste domínio, pelo menos os seguintes riscos devem ser monitorados:

\begin{itemize}

    \item Uma lei não é composta apenas por regras, também inclui elementos que dificilmente podem ser refletidos em fórmulas 
    estáticas (por exemplo, princípios e valores);

    \item Fixar normas num código monolítico é uma forma de tradução do texto legislativo que não permite que as normas se 
    adaptem à evolução da sociedade;

    \item Usar \emph{'linguagens artificiais/formais'} é usar um subconjunto da linguagem natural (Chomsky 2006), 
    o que traz limitações;
    
    \item As normas podem ser intencionalmente mantidas vagas, por exemplo, para implementar um equilíbrio entre diferentes 
    instituições;

    \item Qualquer previsão baseada apenas no passado é inerentemente limitada;
    \item As previsões influenciam os tomadores de decisão e o comportamento humano futuro (Hildebrandt 2021).

\end{itemize}

\subsubsection{Como utilizar a Inteligência Artificial}

O estudo que se resumiu aqui com algumas adaptações ao nosso contexto, \cite{PFPD22}, conclui o seguinte sobre a utilização da IA:

\begin{quoting}
    É possível desenvolver, sob certas condições (a serem definidas), um robusto quadro teórico e empírico 
    jurídico-tecno-linguístico que facilite a tarefa de definir atos normativos de maneira oficial e autoritativa num 
    formato consumível por máquinas (por exemplo, XML) que tenha o mesmo valor legal que o texto em linguagem natural, 
    que por séculos tem sido o meio de escolha para os sistemas jurídicos.

    É evidente que tornar normas jurídicas e éticas consumíveis por máquinas, e não simplesmente legíveis por máquinas, 
    é uma necessidade premente, tornando possível economizar tempo, reagir prontamente a mudanças, permitir a aplicação 
    correta de regulamentos, possibilitar o diálogo máquina a máquina com artefatos digitais (por exemplo, implementar 
    políticas), analisar impactos e tomar decisões rápidas para a economia e a sociedade. 
    Se um quadro sólido de "Lei como Código" for definido e adotado por parlamentos, corpos deliberativos, 
    instituições governamentais e entidades administrativas públicas, podemos economizar quantidades significativas de 
    tempo na implementação prática das normas, evitar erros, monitorar facilmente os efeitos das normas jurídicas, 
    corrigir prescrições ineficazes e desenvolver um ecossistema integrado com as entidades da infosfera digital, 
    preparando a futura interação com robôs, multiagentes, IA, blockchain e contratos inteligentes.
    
    Além disso, os sistemas baseados em IA podem apoiar o processo legislativo na fase de redação para produzir uma 
    regulamentação melhor que evite erros e inconsistências. Essa abordagem economiza dinheiro para as empresas, reduz o 
    número de ações judiciais e diminui os custos de litígios, ao mesmo tempo que torna o cumprimento das regras e o 
    respeito pelo estado de direito mais eficazes.
\end{quoting}    


\subsection{Requisitos da infraestrutura}

\begin{itemize}
\item Arquitetura global da plataforma;
\item Identificação dos serviços que devem compor o sistema;
\item Identificação dos requisitos técnicos de cada serviço;
\item Identificação/previsão das necessidades de processamento, espaço de armazenamento e
conectividade;
\item A Identificação de necessidade de computação em Cloud ou on-premises e respetivos
requisitos;
\item Identificação dos requisitos de interoperabilidade face a sistemas externos (comunicação,
armazenamento e representação dos dados): por exemplo, bases de dados do DRE - INCM e
de Jurisprudência dos tribunais, Ministério da Justiça (MJ), IGFEJ, 
Conselho Superior da Magistratura (CSM).
\end{itemize}


\subsection{Requisitos de sustentabilidade}

Relativamente à sustentabilidade da plataforma a desenvolver podemos apontar alguns divididos por várias categorias.

\subsubsection{Ambientais}

Se se pretender ir de encontro a um dos eixos da comunidade sobre desenvolvimento sustentável:

\begin{description}
    \item[Eficiência Energética:] Utilizar servidores e data centers energeticamente eficientes, preferencialmente com 
    certificação Energy Star;
    \item[Energias Renováveis:] Implementar o uso de energia renovável, como solar ou eólica, para alimentar a infraestrutura;
    \item[Gestão de Resíduos Eletrônicos:] Adotar práticas adequadas para a reciclagem e eliminação de equipamentos eletrónicos;
    \item[Redução de Emissões de CO2:] Monitorar e minimizar a pegada de carbono das operações da plataforma;   
    \item[Design Sustentável de Software:] Desenvolver o software para ser eficiente em termos de recursos, reduzindo a 
    necessidade de hardware potente (\emph{"Green Computing"}).
\end{description}

\subsubsection{Económicos}

A economia associada a uma solução deste tipo é normalmente um fator determinante para a sua sobrevivência.

\begin{description}
    \item[Custo-Benefício:] Garantir que a plataforma seja financeiramente viável e ofereça um bom retorno sobre o investimento.
    Neste contexto, pode ser difícil desenhar um modelo de negócio mas é importante para a sobrevivência da plataforma;
    \item[Escalabilidade:] Desenvolver a plataforma para ser escalável, permitindo o seu crescimento sem necessidade de grandes 
    revisões estruturais;
    \item[Manutenção e Suporte:] Planear para manutenção contínua e suporte técnico, prolongando a vida útil da plataforma, a 
    designada manutenção evolutiva;
    \item[Modelos de Negócio Sustentáveis:] Adotar modelos de negócio que garantam a longevidade financeira da plataforma.
\end{description}    

\subsubsection{Sociais}

Uma plataforma deste tipo é criada para os utilizadores e por isso deve ser criada com eles, envolvendo-os e garantindo a sua
acessibilidade. 

\begin{description}
    \item[Acessibilidade:] Garantir que a plataforma esteja sempre acessível aos 
    utilizadores a quem se destina, eventualmente, seguindo normas como WCAG (Web Content Accessibility Guidelines);
    \item[Ética e Transparência:] Assegurar práticas éticas na gestão de dados e transparência nas políticas de privacidade 
    e segurança;
    \item[Envolvimento da Comunidade:] Envolver a comunidade de utilizadores no desenvolvimento e melhoria contínua da plataforma. 
    A plataforma destina-se a um conjunto de utilizadores o seu constante envolvimento é determinante para o sucesso.
\end{description}

\subsubsection{Técnicos}

Numa realidade em que a tecnologia muda rapidamente e em que obsolescência tecnológica é cada vez mais rápida, atualmente abaixo 
dos 5 anos, é importante garantir algumas condições técnicas.

\begin{description}
    \item[Desempenho:] Otimizar o desempenho da plataforma para garantir tempos de resposta rápidos e operação eficiente;
    \item[Segurança:] Implementar medidas robustas de segurança para proteger dados e garantir a integridade da plataforma;
    \item[Interoperabilidade:] Garantir que a plataforma se possa integrar com outras soluções e tecnologias existentes;
    \item[Atualizações e Melhorias:] Estabelecer um processo contínuo de atualizações e melhorias para manter a plataforma 
    atualizada e eficiente.
\end{description}

\subsubsection{Gestão}

A sustentabilidade depende também de alguns processos de gestão continuados.

\begin{description}
    \item[Conformidade Legal:] Assegurar que a plataforma esteja em conformidade com todas as regulamentações e leis aplicáveis
    (RGPD, Política de Segurança, etc);
    \item[Política de Sustentabilidade:] Desenvolver e seguir uma política de sustentabilidade clara que guie as operações da 
    plataforma;
    \item[Controlo:] Implementar um sistema para monitorar e relatar o desempenho em termos de sustentabilidade 
    regularmente.
\end{description}

Estes requisitos ajudam a garantir que uma plataforma digital não apenas funcione de forma eficiente, mas também 
contribua positivamente para o meio ambiente, a sociedade e a economia, e que tenha a longevidade esperada e proporcional ao 
investimento que é feito.



