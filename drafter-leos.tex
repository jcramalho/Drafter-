\chapter{Protótipo/Prova de Conceito}
\label{prototipo}

Neste capítulo, descrevem-se os trabalhos conducentes à criação da prova de conceito do sistema a desenvolver. 
Começa-se com uma pequena introdução contextual para logo entrar nos detalhes técnicos.

\section{Introdução}

Além dos relatórios produzidos, será desenvolvida uma prova de conceito, a uma escala
reduzida, que permitirá elucidar alguns dos requisitos e, provavelmente, levantar novos requisitos ainda
não especificados.

A prova de conceito a desenvolver será composta pelas seguintes atividades e respetivos
resultados:

\begin{itemize}
\item Adoção do software open source LEOS (Legislation Editing Open Software ), como base da
solução, instalação e disponibilização online;
\item Colheita de um subconjunto de legislação do DRE;
\item Colheita de algumas bases de dados de jurisprudência dos tribunais;
\item Especificação de um modelo ontológico para a legislação colhida (baseada no trabalho já
realizado pelo EPO no ELI);
\item Processamento/Mineração, usando técnicas de NLP (Natural Language Processing ), da
legislação colhida para extração de dados para o povoamento da ontologia especificada;
\item Disponibilização da ontologia através de um motor de gestão de bases de dados orientadas a
grafos online;
\item Disponibilização de uma interface de pesquisa baseada em SPARQL que permitirá navegar na
ontologia;
\item Integração do LEOS com a base de dados ontológica: como suporte à edição de legislação;
\item  (Possibilidade) Identificar os vários tipos de documentos legislativos e estudar a hipótese de
aplicar técnicas de Machine Learning (ML) para gerar automaticamente conteúdo novo no
documento que está a ser editado.
\end{itemize}

\section{LEOS: Legislation Editing Open Software}

O LEOS é um projeto no âmbito da iniciativa \emph{"Interoperable Europe"} da Comissão Europeia para uma política reforçada de 
interoperabilidade do setor público, financiado pelo Programa Digital \emph{Europe (DIGITAL)} e criado para atender à 
necessidade da administração pública e das Instituições Europeias de gerar projetos de legislação em formato XML jurídico.

O projeto LEOS concentra-se em apoiar o co-desenvolvimento, co-design e co-implementação de um "ecossistema de Tecnologias de 
Informação (TI) centrado num LEOS aumentado".

O LEOS foi criado para abordar a modernização e transformação digital da elaboração e revisão de legislação nas 
Instituições da UE, agências e órgãos da UE e Estados-Membros.

Esta plataforma garante que o conteúdo elaborado pelos utilizadores siga as diretrizes de redação, oferecendo recursos como a
aplicação de estruturas de documento pré-definidas, layout pré-definido e regras de numeração. 
Tudo isso para garantir que o autor se possa focar na elaboração do texto e muito menos na gestão do layout (ou verificação). 
Para facilitar a colaboração online eficiente, o LEOS também possui outros recursos como comentários, sugestões, 
controle de versão, edição colaborativa, etc.


\section{Stack Tecnológica}

O LEOS é um ecossitema de serviços relativamente complexo que se passa a descrever.

\subsection{Frontend: interface com o utilizador}

O frontend da plataforma LEOS foi construído usando a framework Vaadin 8 e o AngularJS. 

O Vaadin \footnote{\url{https://vaadin.com/vaadin-8}} é uma framework Java usada para a criação de interfaces Web. 
Esta framework permite gerar vários componentes UI (\emph{"User Interface"}) usando código em JAVA.
A versão 8 em específico foi descontinuada em 21 de Fevereiro, 2022. 
Esta versão já não recebe suporte por quem a desenvolveu e mantinha.

No desenvolvimento do sistema português é preciso acautelar esta situação.
Confirmar junto da equipa de desenvolvimento do LEOS quais os planos para a substituição deste componente ou, em caso 
da inexistência desses planos prever o seu desenvolvimento numa tecnologia atual.

O AngularJS \footnote{\url{https://angularjs.org/}} é uma framework JavaScript que permite criar interfaces visuais reativas. 
No LEOS utilizou-se esta framework para se desenvolver o cliente de anotações.


\subsection{Backend}

O backend da Plataforma LEOS consiste numa aplicação criada com a framework Spring \footnote{\url{https://spring.io/projects/spring-boot}}. 
Esta framework, desenvolvida em JAVA, é reconhecida pela sua estabilidade e fácil manutenção.

De modo a estabelecer a conexão com a base de dados, a aplicação Spring utiliza o módulo Spring Data JPA \footnote{\url{https://spring.io/projects/spring-data-jpa}}. 
Este módulo cria uma camada de abstração sobre a API de persistência do JAVA (JPA), permitindo armazenar e recuperar 
informação de bases de dados relacionais eficientemente.

Para permitir criar abstrações da base de dados, o LEOS usa o Hibernate.
O Hibernate\footnote{\url{https://hibernate.org/}} é uma framework de \emph{"object-relational mapping"} (ORM) para a linguagem Java. 
Esta framework permite realizar o mapeamento entre os modelos orientados a objetos e as bases de dados relacionais tornando 
transparente para o programador a sua implementação.

Para servir o backend, o LEOS utiliza um servidor web Apache Tomcat para receber e responder aos pedidos HTTP. 
O apache Tomcat é também um Servlet container para aplicações java. 
Na plataforma LEOS, este servidor é responsável por servir a aplicação Spring.

Por último, a framework Mockito\footnote{\url{https://site.mockito.org/}} é utilizada para a realização de testes unitários 
para a linguagem de programação JAVA.

\subsection{Persistência}
 
O LEOS utiliza uma base de dados relacional para guardar toda a informação necessária ao seu funcionamento normal. 
Estas informações incluem dados relacionados com utilizadores, configurações, metadados de documentos, etc. 
Por omissão, é utilizada a base de dados H2\footnote{\url{https://www.h2database.com/html/main.html}} cujas caraterísticas 
principais são ser rápida, fácil de usar, de dimensão pequena, poder ser instanciada num servidor ou em memória e 
poder ser gerida via web.

O LEOS guarda as anotações aos documentos numa base de dados à parte. 
Por omissão, é utilizada uma base de dados H2 em memória.

Além da H2, LEOS utiliza um servidor CMIS\footnote{\url{https://www.oasis-open.org/standard/cmisv1-1/}} 
(\emph{"Content Management Interoperability Services"}) para guardar documentos. 
CMIS é um standard OASIS que permite partilha de informação entre diferentes sistemas de gestão de informação. 
De modo a criar este servidor o LEOS utiliza a  ferramenta OpenCMIS do projeto Apache Chemistry.

\subsection{Sumário}

Concluindo, o LEOS assenta num conjunto de tecnologias diversos, com alguma complexidade e com alguns 
riscos associados identificados em cima: Vaadin 8, AngularJS, JAVA Spring, Spring Data JPA, Hibernate, Apache Tomcat,
Mockito, H2 e CMIS.

\section{Configuração do idioma: português}

O LEOS foi desenvolvido utilizando a norma \texttt{i18n}.

A internacionalização (i18n) é o processo de projetar e desenvolver um produto de software para que possa ser adaptado para 
utilizadores de diferentes culturas e idiomas.

A internacionalização não envolve apenas permitir diferentes idiomas, mas também adaptar o software para aceitar diferentes 
formas de dados e configurações para corresponder aos costumes locais e processá-los corretamente.

O Grupo W3C define a internacionalização como:
\begin{quoting}
A internacionalização é o design e o desenvolvimento de um produto, aplicação ou conteúdo que permite fácil 
adaptação a públicos-alvo de cultura, região ou idioma diferentes.
\end{quoting}



Neste contexto, o LEOS é facilmente adaptável a utilizadores de culturas e línguas diferentes. 

Esta norma garante que os dados associados às traduções da plataforma ficam separados do código principal, 
tornando o processo de adaptação do LEOS para novas línguas flexível e eficiente.

Para se configurar o idioma português para o LEOS foi necessário criar um novo ficheiro JSON com as traduções portuguesas dos 
textos da interface gráfica do LEOS. 

Na pasta \verb|\leos\modules\ui-angular\src\assets\i18n\| já se encontram ficheiros JSON para a lingual Inglesa e Francesa. 
Para gerar a versão Portuguesa, traduziu-se automaticamente todas os textos contidos no ficheiro original em Inglês 
(\texttt{en.json}), 
um processo seguido de revisão manual. 
Desta forma foi criado o ficheiro \texttt{pt.json}, o qual foi depois colocado na mesma pasta que os originais.

De seguida, foi necessário alterar o ficheiro de configuração presente na pasta \\
\verb|\leos\modules\ui-angular\src\config\global.ts|, de modo a incluir o ficheiro com traduções portuguesas no LEOS. 
Esta configuração consistiu em adicionar a string \texttt{"pt"} à lista das \emph{languages}, que faz parte do objeto i18n. 
Foi ainda alterado o atributo \texttt{defaultLanguage} de "en" para "pt", no entanto, esta alteração não demonstrou qualquer 
alteração no comportamento do sistema, dado que o inglês continuou a ser a língua por omissão.

\begin{Verbatim}[frame=single, numbers=left, fontsize=\scriptsize]
i18n: {
    i18nService: {
      defaultLanguage: 'pt',
      languages: ['pt', 'en', 'fr'],
    }, ... 
 },
\end{Verbatim}


\section{Modelos/Templates}

\subsection{Adição de novos modelos}

De momento, a criação de novos modelos para a plataforma LEOS é um processo manual sendo necessário a edição/criação de vários 
ficheiros e a sua colocação em locais estratégicos da plataforma.

Vai-se exemplificar este processo com um caso de estudo simples de uma lei.
Ou seja, assuma-se que se pretende gerar um novo modelo para um ato normativo com a tipologia de Lei.
 
No LEOS, os modelos de atos normativos são representados no formato \texttt{Akoma Ntoso (AKN)} já apresentado anteriromente. 
Desta forma, e para iniciar o processo, é necessário gerar um ficheiro no formato AKN para representar a estrutura de uma Lei. 
De momento o LEOS não fornecesse qualquer ferramenta para gerar este tipo de ficheiros pelo que estes têm de ser gerados 
manualmente.

Depois de gerado, o ficheiro AKN deve ser colocado na pasta dos modelos do LEOS: 
\texttt{\url{LEOS_dir/tools/repositor/server/src/main/resources/leos/templates}}.

Cada modelo associado a um dado ato normativo deve ser acompanhado de um outro ficheiro modelo para gerar a página de capa 
desse ato normativo. Este segundo ficheiro também deve ser colocado na pasta dos modelos.
 
A seguir é necessário adicionar uma nova entrada, para cada modelo, em duas tabelas da base de dados (do LEOS) 
para que o LEOS consiga aceder aos novos modelos.

A título de exemplo, eis as \emph{queries} SQL para um novo modelo:

\begin{Verbatim}[frame=single, numbers=left, fontsize=\scriptsize, commandchars=\\\{\}]
    INSERT INTO CONFIG (NAME,OBJECT_ID,AUDIT_C_BY,AUDIT_C_DATE,
                AUDIT_LAST_M_BY,AUDIT_LAST_M_DATE,LANGUAGE,CATEGORY_ID) 
            VALUES ('LEI-01','0','admin/admin',
                   to_timestamp('22-01-21 08:25:40.328000000',
                   'DD-MM-RR HH24:MI:SSXFF'),'admin/admin',
                   to_timestamp('22-01-21 08:25:40.328000000',
                   'DD-MM-RR HH24:MI:SSXFF'),'PT',
                   (SELECT id from CONFIG_CATEGORIES WHERE 
                     CATEGORY_CODE='TEMPLATE_BILL'));
\end{Verbatim}

e

\begin{Verbatim}[frame=single, numbers=left, fontsize=\scriptsize, commandchars=\\\{\}]
    INSERT INTO CONFIG_VERSION (CONFIG_ID,VERSION_LABEL,VERSION_SERIES_ID,
        VERSION_TYPE,IS_LATEST_MAJOR_VERSION,IS_LATEST_VERSION,
        IS_MAJOR_VERSION,IS_VERSION_SERIES_CHECKED_OUT,AUDIT_C_BY,
        AUDIT_C_DATE,AUDIT_LAST_M_DATE,AUDIT_LAST_M_BY,IS_IMMUTABLE) 
      VALUES ((SELECT id from CONFIG WHERE NAME='LEI-01'),'1.0','1',null,1,1,
        1,0,'admin/admin',to_timestamp('30-03-23 07:38:37.451000000',
        'DD-MM-RR HH24:MI:SSXFF'),to_timestamp('30-03-23 07:38:37.496000000',
        'DD-MM-RR HH24:MI:SSXFF'),'admin/admin',0);
\end{Verbatim}

Para finalizar a configuração, é necessário editar o ficheiro \texttt{catalog.xml}, 
presente na mesma pasta dos modelos, de modo a adicionar os modelos gerados ao catálogo de modelos 
para que estes fiquem disponíveis no LEOS. 
Este ficheiro define a estrutura em árvore apresentada no auxiliar de criação de documentos. 
É aqui que devemos colocar a referência do novo modelo para que ele seja visível na árvore (fig.\ref{fig-catalog}).

\figura{fig-catalog}{./imagens/catalog2.png}{LEOS: árvore de tipologias disponíveis}{width=10cm}
  
Exemplo:
 
\begin{Verbatim}[frame=single, numbers=left, fontsize=\scriptsize, commandchars=\\\{\}]
<?xml version="1.0" encoding="UTF-8" ?>
<catalog lang="PT">
    <item type="CATEGORY" id="c1" enabled="true" key="LAW_INITIATIVE">
        <names>
            <name lang="PT">Atos Normativos</name>
        </names>
        <item type="CATEGORY" id="c1.1" enabled="true" 
              key="ATOS_NORMATIVOS">
            <names>
                <name lang="PT">Lei</name>
                <name lang="EN">Law</name>            </names>
            <item type="TEMPLATE" id="LEI-PR-01;LEI-01" enabled="true" 
                  key="lei-01">
                <names>
                    <name lang="PT">01 - Modelo de Lei</name>
                    <name lang="EN">01 – Law Template</name>
                </names>
                <languages>
                    <language lang="PT">Portuguese</language>
                    <language lang="EN">English</language>
                </languages>
            </item>
        </item>
    </item>
    ... 
</catalog>
\end{Verbatim} 


\section{Interface}

Nesta secção, apresentam-se vários exemplos da interface do LEOS configurada para a realidade portuguesa.

Criaram-se modelos muito simples para as tipologias de atos normativos apenas para efeitos de exemplificação.
 
\figura{fig-catalog}{./imagens/leos_lista02.png}{LEOS: resultados da pesquisa de propostas}{width=15cm}

\figura{fig-catalog}{./imagens/leos_lei.png}{LEOS: modelo de uma lei na plataforma}{width=10cm}