\chapter{Introdução}

Esta atividade tem por objetivo a definição dos requisitos de uma plataforma
que auxilie na produção de atos normativos, com recurso a mecanismos de 
Inteligência Artificial (IA), em concordância
com os elementos apurados na atividade anterior do estudo 
(as melhores práticas existentes nos
sistemas de informação utilizados para apoio à redação legislativa), 
que suportem os requisitos técnicos do procedimento a lançar para a execução 
da plataforma e que deve cumprir com as caraterísticas enumeradas a seguir.

\section{Requisitos}

\subsection{Requisitos funcionais de alto nível}

\begin{itemize}
\item Deteção automática de cumprimento de normas de produção de atos normativos;
\item Observação de interpretações firmadas em jurisprudência sobre normas;
\item Apoio à elaboração de tarefas de avaliação normativa;
\item Automatização dos processos de avaliação legislativa para textos preparados na plataforma;
\item Verificação e validação das referências normativas e legais identificadas nos textos preparados
na plataforma (verificação da existência das normas invocadas);
\item Pesquisa e identificação automática de legislação e jurisprudência;
\item Verificação semântica das normas invocadas.
\end{itemize}

\subsection{Requisitos funcionais}

\begin{itemize}
\item Descrição dos workflows para a criação e gestão de atos normativos;
\item (Possibilidade) Descrição destes processos em BPMN;
\item (Possibilidade) Descrição do ciclo de vida dos documentos na plataforma.
\end{itemize}

\subsection{Requisitos de interoperabilidade}

\begin{itemize}
\item Integração com fontes primárias fundamentais, designadamente bases de dados, com toda a
legislação e atos normativos, para apoio à redação legislativa e normativa: p.e., Diário da
República;
\item Integração com fontes primárias de jurisprudência: p. ex., ECLI;
\item Integração com entidades parceiras identificadas para o fornecimento de dados;
\item Interoperabilidade técnica: protocolos de comunicação, API de dados REST ou Web Service ;
\item Interoperabilidade Sintática: formato de importação e exportação de dados; deverá ser baseado
em XML e seguir normas internacionais (Akoma Ntoso XML format, uma norma OASIS para
documentos legislativos);
\item Interoperabilidade Semântica: representação semântica dos dados, ontologias OWL, utilização
do ELI e da ontologia associada;
\end{itemize}

\subsection{Identificação dos mecanismos de IA e das ferramentas conexas a usar no desenvolvimento
da plataforma}

\begin{itemize}
\item Utilização de mecanismos de Processamento de Linguagem Natural (PLN) e Mineração de
Texto, para extração de representação e significados dos textos disponíveis na base de dados
da plataforma;
\item Utilização de aprendizagem pela máquina (Machine Learning ) para incrementar a precisão do
sistema (processo de validação pelo utilizador/Configurador);
\item Definição de regras de mapeamento para aumentar a precisão das árvores de decisão
adotadas;
\item Definição de estratégia de redes neuronais para efeitos de utilização de sistemas de previsão,
designadamente na identificação de legislação e jurisprudência;
\item Identificação dos algoritmos mais adequados e o seu futuro desenvolvimento;
\item Identificação das necessidades de treino do sistema de IA.
\end{itemize}


\subsection{Requisitos da infraestrutura}

\begin{itemize}
\item Arquitetura global da plataforma;
\item Identificação dos serviços que devem compor o sistema;
\item Identificação dos requisitos técnicos de cada serviço;
\item Identificação/previsão das necessidades de processamento, espaço de armazenamento e
conectividade;
\item A Identificação de necessidade de computação em Cloud ou on-premises e respetivos
requisitos;
\item Identificação dos requisitos de interoperabilidade face a sistemas externos (comunicação,
armazenamento e representação dos dados): por exemplo, bases de dados do DRE - INCM e
de Jurisprudência dos tribunais, Ministério da Justiça (MJ), IGFEJ, 
Conselho Superior da Magistratura (CSM).
\end{itemize}


\section{Protótipo/Prova de Conceito}

Além dos relatórios produzidos, será desenvolvida uma prova de conceito, a uma escala
reduzida, que permitirá elucidar alguns dos requisitos e, provavelmente, levantar novos requisitos ainda
não especificados.

A prova de conceito a desenvolver será composta pelas seguintes atividades e respetivos
resultados:

\begin{itemize}
\item Adoção do software open source LEOS (Legislation Editing Open Software ), como base da
solução, instalação e disponibilização online;
\item Colheita de um subconjunto de legislação do DRE;
\item Colheita de algumas bases de dados de jurisprudência dos tribunais;
\item Especificação de um modelo ontológico para a legislação colhida (baseada no trabalho já
realizado pelo EPO no ELI);
\item Processamento/Mineração, usando técnicas de NLP (Natural Language Processing ), da
legislação colhida para extração de dados para o povoamento da ontologia especificada;
\item Disponibilização da ontologia através de um motor de gestão de bases de dados orientadas a
grafos online;
\item Disponibilização de uma interface de pesquisa baseada em SPARQL que permitirá navegar na
ontologia;
\item Integração do LEOS com a base de dados ontológica: como suporte à edição de legislação;
\item  (Possibilidade) Identificar os vários tipos de documentos legislativos e estudar a hipótese de
aplicar técnicas de Machine Learning (ML) para gerar automaticamente conteúdo novo no
documento que está a ser editado.
\end{itemize}

