\chapter{Requisitos}
\label{requisitos}

Na proposta subjacente ao contrato ao abrigo do qual se redige este relatório, constavam vários grupos de requisitos.
Neste capítulo, contextualiza-se cada um deles de acordo com o protótipo (cap. \ref{prototipo}) instalado, parametrizado 
e desenvolvido.

Relativamente ao protótipo, era um requisito que este se baseasse na plataforma LEOS (\emph{"Legislation Editing Open Software"}).
O LEOS tem já desenvolvidos e implementados muitos dos requisitos, carecendo de uma adaptação à realidade portuguesa.
Ao longo deste capítulo, faz-se uma associação de cada um dos requisitos a uma ou mais funcionalidades da plataforma LEOS,
indicando as parametrizações que será necessário fazer. Para os casos em que a associação não seja possível indicam-se as 
funcionalidades a desenvolver.

A criação desta nova plataforma pressupõe também uma ligação automática (interoperabilidade) a outro sistema onde estaria 
a jurisprudência nacional. Para isso, seria necessário que existisse uma normalização sobre a jurisprudência e que esta fosse 
disponibilizada numa API de dados. Algo que poderá acontecer num futuro próximo.

\section{Requisitos}

Nas subsecções seguintes descrevem-se os conceitos associados a cada requisito, se estão ou não presentes no LEOS e,
o que será preciso fazer caos não estejam.

\subsection{Requisitos funcionais}

Alguns destes requisitos estão relacionados com políticas sobre a adoção de normas e a especificação e 
criação de processos, no entanto,
ao colocarmos o LEOS como base, algumas destas decisões já foram tomadas e parte deste trabalho já está feita.

\subsubsection{Deteção automática de cumprimento de normas de produção de atos normativos}

O LEOS precisa de ser parametrizado com os modelos estruturais dos atos normativos que se pretendem criar.
No capítulo anterior, apresentou-se uma análise prévia necessária à criação destes modelos. 
O passo seguinte, será 
especificar esses modelos em XML numa linguagem específica desenvolvida para o efeito \cite{VPCB2018}.
Apresentam-se exemplos no capítulo referente à configuração da plataforma \ref{modelos}.

Esta especificação de cada um dos modelos pode e deve incluir os requisitos estruturais que devem ser observados na 
produção de atos normativos. Ao fazê-lo, o LEOS ou outra plataforma que os venha a utilizar pode verificar automaticamente 
o cumprimento dos requisitos/normas.

\subsubsection{Observação de interpretações firmadas em jurisprudência sobre normas}

Este requisito pressupõe a verificação de algumas condições estruturais e semânticas.
As condições estruturais podem e devem estar refletidas no modelo criado em AKN (\ref{akn}) para a tipologia.
As condições semânticas podem ser divididas em duas: estáticas e dinâmicas.
Ambas terão de ser especificadas no momento da definição do modelo para a tipologia. 
A semântica estática pode ficar definida no modelo em AKN ou ainda numa estensão a este modelo criada na plataforma LEOS.
A semântica dinâmica, depende de valores que serão introduzidos aquando da criação do documento e a sua verificação terá 
de ser garantida usando uma linguagem extra de regras ou materializadas no código da aplicação.

\subsubsection{Apoio à elaboração de tarefas de avaliação normativa}

As tarefas de avaliação normativa pretendem avaliar o impacto da nova legislação que está a ser criada.
É algo externo ao ato normativo e para o qual é necessário a interpretação e análise humana.
O apoio será no registo da informação que possa conduzir a um valor final de impacto. 

O LEOS não possui qualquer suporte à elaboração destas tarefas.
Para isso, será necessário desenvolver um componente de raiz, que poderá estar integrado no LEOS ou comunicar com este 
sendo um serviço externo.

O desenvolvimento deste componente carece de uma especificação cuidada de requisitos.

\subsubsection{Automatização dos processos de avaliação legislativa para textos preparados na plataforma}

Relacinonado com o ponto anterior. Estas funcionalidades deverão fazer parte do novo componente a desenvolver.

\subsubsection{Verificação e validação das referências normativas e legais identificadas nos textos preparados
na plataforma (verificação da existência das normas invocadas)}

Como já foi referido em cima, para o cumprimento deste requisito é necessário dispôr de uma base de dados com a 
produção jurídico-normativa de Portugal.

Para isso, é necessário garantir a adoção de várias regras/normas:

\begin{itemize}
    \item É preciso garantir uma forma única de referenciar os atos normativos, seja ela o ELI 
    ("\emph{European Legislation Identifier}") \cite{ELI}, ou simplesmente aquilo a que estamos habituados em Portugal, ou 
    seja, uma referência composta por tipologia, número de série e ano; 

    \item Garantir a presença \emph{online}, de um repositório de atos normativos permanentemente atualizado e acessível;
    \item Ter uma API ("\emph{Application Program Interface}") bem definida sobre o repositório anterior que permita a 
    integração automática e a interoperabilidade com outros sistemas, por exemplo, o que se pretende desenvolver.
\end{itemize}

Devido a inexistência destas normas, podemos avançar com um pequeno protótipo próprio mas que terá sempre as desvantagens de 
estar fora dos canais oficiais e de sofrer de permanente desatualização.


\subsubsection{Pesquisa e identificação automática de legislação e jurisprudência}

Os requisitos descritos no ponto anterior são também válidos neste ponto.
Havendo um repositório de legislação que obedeça a um conjunto de regras transversais, basta incluir uma funcionalidade de 
pesquisa na sua API.

Relativamente à jurisprudência, a situação é semelhante à da legislação, é necessário garantir a adoção de várias regras/normas:

\begin{itemize}
    \item É preciso garantir uma forma única de referenciar os processos. Para isto já existe o ECLI 
    ("\emph{European Case Law Identifier}") \cite{ECLI}. É preciso promover/decretar a sua adoção à escala nacional; 

    \item Garantir uma normalização a nível dos metadados à escala nacional, todas as instituições produtoras de 
   jurisprudência deverão produzi-la na mesma forma, o que não se passa atualmente;

    \item Garantir a presença \emph{online}, de um repositório de jurisprudência permanentemente atualizado e acessível;

    \item Ter uma API ("\emph{Application Program Interface}") bem definida sobre o repositório anterior que permita a 
    integração automática e a interoperabilidade com outros sistemas, por exemplo, o que se pretende desenvolver.
\end{itemize}

À semelhança do ponto anterior, devido a inexistência destas normas, podemos avançar com um pequeno protótipo próprio mas que terá sempre as desvantagens de 
estar fora dos canais oficiais e de sofrer de permanente desatualização.

\subsubsection{Verificação semântica das normas invocadas}

Para suportar este requisito será preciso ir mais além da criação dos modelos AKN para as tipologias.
Será necessário criar um modelo semântico para cada uma, aquilo que se designa por ontologia.

Neste contexto, uma ontologia pode definir-se como uma especificação formal de conhecimento de um determinado domínio bem 
caraterizado e que inteligível quer para máquinas quer para humanos.

Há várias linguagens que permitem a criação de ontologias com níveis diferentes de aplicabilidade.
Aquela que melhor se adapta à complexidade do contexto jurídico-normativo é a "\emph{Ontology Web Language} (OWL)" 
\cite{owl,owl2}.

A OWL é uma linguagem de anotação desenhada para especificar semântica, publicar e partilhar dados. 
Foi desenvolvida pelo \emph{World Wide Web Consortium} (W3C).

Um ontologia OWL é um conjunto de conceitos, seus atributos e relações entre eles. 
Adicionalmente, permite a especificação de regras semânticas via axiomas.

As ontologias são uma maneira de modelar o conhecimento de uma forma estruturada, permitindo que diferentes sistemas 
compreendam e usem esses dados de forma interoperável.

A linguagem OWL é baseada em \emph{Description Logic} (DL), que fornece uma base formal para o raciocínio automatizado. 
Isso significa que a partir de uma ontologia de base a máquina consegue inferir novos fatos a partir dos dados existentes, 
aumentar automaticamente o seu conhecimento e com isso 
melhorar a busca de informação e facilitar a integração de dados de diversas fontes.
A linguagem tem vários níveis de complexidade deixando a quem a utiliza a decisão de qual o nível que pretende usar. 
Desta forma, pode-se dizer que tudo se pode especificar em OWL.

\subsubsection{Descrição dos workflows para a criação e gestão de atos normativos}

O Ministério da Justiça possui já uma ferramenta própria para automatização de workflows. 
Se for possível colocar o novo sistema a dialogar com essa ferramenta consegue-se cumprir este requisito e 
poupar imenso tempo de desenvolvimento.


\subsection{Requisitos de interoperabilidade}

Hoje em dia, falar de interoperabilidade é quase um requisito mas muitos desconhecem que esta é complexa 
e pode ser considerada em vários e diferentes níveis.

Da bibliografia, interoperabilidade é a capacidade de um sistema (informatizado ou não) de comunicar de 
forma transparente (ou o mais próximo disso) com outro sistema (semelhante ou não).

A nossa Agência para a Modernização Administrativa (AMA)  definiu um modelo para a interoperabilidade na 
Administração Pública (AP):
\begin{itemize}
\item Neste contexto, a interoperabilidade consiste na capacidade das organizações interagirem 
e agirem em prol de benefícios comuns, através de comunicação e partilha de informação e conhecimento;
\item O modelo foi baseado na \emph{Framework Europeia de Interoperabilidade} \cite{EIF}, criada pela Comissão Europeia;
\item Está organizado em quatro camadas: legal, organizacional, técnica e semântica.
\end{itemize}

Apesar de independentes, as quatro camadas são interdependentes, sendo que para se conseguir atingir a interoperabilidade 
semântica, as outras terão de estar contempladas (neste projeto é importante atingir o patamar semântico).

\begin{itemize}
\item Integração com fontes primárias fundamentais, designadamente bases de dados, com toda a
legislação e atos normativos, para apoio à redação legislativa e normativa: p.e., Diário da
República;
\item Integração com fontes primárias de jurisprudência: p. ex., ECLI;
\item Integração com entidades parceiras identificadas para o fornecimento de dados;
\item Interoperabilidade técnica: protocolos de comunicação, API de dados REST ou Web Service ;
\item Interoperabilidade Sintática: formato de importação e exportação de dados; deverá ser baseado
em XML e seguir normas internacionais (Akoma Ntoso XML format, uma norma OASIS para
documentos legislativos);
\item Interoperabilidade Semântica: representação semântica dos dados, ontologias OWL, utilização
do ELI e da ontologia associada;
\end{itemize}

\subsection{Identificação dos mecanismos de IA e das ferramentas conexas a usar no desenvolvimento
da plataforma}

\begin{itemize}
\item Utilização de mecanismos de Processamento de Linguagem Natural (PLN) e Mineração de
Texto, para extração de representação e significados dos textos disponíveis na base de dados
da plataforma;
\item Utilização de aprendizagem pela máquina (Machine Learning ) para incrementar a precisão do
sistema (processo de validação pelo utilizador/Configurador);
\item Definição de regras de mapeamento para aumentar a precisão das árvores de decisão
adotadas;
\item Definição de estratégia de redes neuronais para efeitos de utilização de sistemas de previsão,
designadamente na identificação de legislação e jurisprudência;
\item Identificação dos algoritmos mais adequados e o seu futuro desenvolvimento;
\item Identificação das necessidades de treino do sistema de IA.
\end{itemize}


\subsection{Requisitos da infraestrutura}

\begin{itemize}
\item Arquitetura global da plataforma;
\item Identificação dos serviços que devem compor o sistema;
\item Identificação dos requisitos técnicos de cada serviço;
\item Identificação/previsão das necessidades de processamento, espaço de armazenamento e
conectividade;
\item A Identificação de necessidade de computação em Cloud ou on-premises e respetivos
requisitos;
\item Identificação dos requisitos de interoperabilidade face a sistemas externos (comunicação,
armazenamento e representação dos dados): por exemplo, bases de dados do DRE - INCM e
de Jurisprudência dos tribunais, Ministério da Justiça (MJ), IGFEJ, 
Conselho Superior da Magistratura (CSM).
\end{itemize}


\subsection{Requisitos de sustentabilidade}



