\chapter{Tipologias de atos normativos}
\label{tipologias}

A plataforma que se pretnde criar deverá ter modelos pré-criados para todas as tipologias de atos normativos 
que se venham a suportar. No contexto legístico português há dezenas de tipologias. No âmbito deste trabalho e devido a 
restrições temporais, foi necessário reduzir a um subconjunto mas sihnificativo e representativo do que se pretende.

Consultou-se a entidade adjudicante que designou como prioritárias a lei, o decreto-regulamentar, o decreto, a portaria e o despacho normativo 
(publicado na 2.a série do Diário da República) e, em certos casos, a resolução do Conselho de Ministros.

\section{Lei}

Acto legislativo, emanado pela Assembleia da República, no exercício da sua função legislativa, ao abrigo do artigo 164.º e 165.º da Constituição da República Portuguesa
"As leis da Assembleia da República obedecem ao formulário seguinte:
«A Assembleia da República decreta, nos termos da alínea... do artigo 161.º da Constituição, o seguinte:
(Segue-se o texto.)» Tratando-se de lei constitucional ou orgânica, deve mencionar-se expressamente o termo correspondente, na parte final da fórmula. Tratando-se de resoluções de aprovação de tratados ou acordos internacionais, o texto é composto do seguinte modo:
«Aprovar (para ratificação, no caso dos tratados) o... (segue-se a identificação do tratado ou do acordo internacional em forma simplificada, com indicação da matéria a que respeita, do local e data da assinatura, sendo o teor do respetivo instrumento publicado em anexo).»"

L n.º 74/98, de 11 de novembro
1.ª Série
n.º 74
11/11/1998
Assembleia da República
Lei n.º 74/98, de 11 de novembro
Publicação, identificação e formulário dos diplomas
A Assembleia da República decreta, nos termos da alínea c) do artigo 161.º da Constituição, para valer como lei geral da República, o seguinte:
v. diploma
v. diploma

