\chapter{Requisitos}
\label{requisitos}

Na proposta subjacente ao contrato ao abrigo do qual se redige este relatório, constavam vários grupos de requisitos.
Neste capítulo, contextualiza-se cada um deles de acordo com o protótipo (cap. \ref{prototipo}) instalado, parametrizado 
e desenvolvido.

Relativamente ao protótipo, era um requisito que este se baseasse na plataforma LEOS (\emph{"Legislation Editing Open Software"}).
O LEOS tem já desenvolvidos e implementados muitos dos requisitos, carecendo de uma adaptação à realidade portuguesa.
Ao longo deste capítulo, faz-se uma associação de cada um dos requisitos a uma ou mais funcionalidades da plataforma LEOS,
indicando as parametrizações que será necessário fazer. Para os casos em que a associação não seja possível indicam-se as 
funcionalidades a desenvolver.

\section{Requisitos}

Nas subsecções seguintes descrevem-se os conceitos associados a cada requisito, se estão ou não presentes no LEOS e,
o que será preciso fazer caos não estejam.

\subsection{Requisitos funcionais}

Alguns destes requisitos estão relacionados com políticas sobre a adoção de normas e a especificação e 
criação de processos, no entanto,
ao colocarmos o LEOS como base, algumas destas decisões já foram tomadas e parte deste trabalho já está feita.

\subsubsection{Deteção automática de cumprimento de normas de produção de atos normativos}

O LEOS precisa de ser parametrizado com os modelos estruturais dos atos normativos que se pretendem criar.
No capítulo anterior, apresentou-se uma análise prévia necessária à criação destes modelos. O passo seguinte, será 
especificar esses modelos em XML numa linguagem específica desenvolvida para o efeito (\ref{???}).
Apresentam-se exemplos mais à frente \ref{???}.

Esta especificação de cada um dos modelos pode e deve incluir os requisitos estruturais que devem ser observados na 
produção de atos normativos. Ao fazê-lo, o LEOS ou outra plataforma que os venha a utilizar pode verificar automaticamente 
o cumprimento dos requisitos/normas.

\subsubsection{Observação de interpretações firmadas em jurisprudência sobre normas}

\subsubsection{Apoio à elaboração de tarefas de avaliação normativa}

\subsubsection{Automatização dos processos de avaliação legislativa para textos preparados na plataforma}

\subsubsection{Verificação e validação das referências normativas e legais identificadas nos textos preparados
na plataforma (verificação da existência das normas invocadas)}

\subsubsection{Pesquisa e identificação automática de legislação e jurisprudência}

\subsubsection{Verificação semântica das normas invocadas}

\subsubsection{Descrição dos workflows para a criação e gestão de atos normativos}

(Possibilidade) Descrição destes processos em BPMN

(Possibilidade) Descrição do ciclo de vida dos documentos na plataforma.


\subsection{Requisitos de interoperabilidade}

Hoje em dia, falar de interoperabilidade é quase um requisito mas muitos desconhecem que esta é complexa 
e pode ser considerada em vários e diferentes níveis.

Da bibliografia, interoperabilidade é a capacidade de um sistema (informatizado ou não) de comunicar de 
forma transparente (ou o mais próximo disso) com outro sistema (semelhante ou não).

A nossa Administração Pública (AP) definiu um modelo para a interoperabilidade:
\begin{itemize}
\item Neste contexto, a interoperabilidade consiste na capacidade das organizações interagirem 
e agirem em prol de benefícios comuns, através de comunicação e partilha de informação e conhecimento;
\item O modelo foi baseado na \emph{Framework Europeia de Interoperabilidade}, criada pela Comissão Europeia;
\item Está organizado em quatro camadas: legal, organizacional, técnica e semântica.
\end{itemize}

Apesar de independentes, as quatro camadas são interdependentes, sendo que para se conseguir atingir a interoperabilidade 
semântica, as outras terão de estar contempladas (neste projeto é importante atingir o patamar semântico).


\begin{itemize}
\item Integração com fontes primárias fundamentais, designadamente bases de dados, com toda a
legislação e atos normativos, para apoio à redação legislativa e normativa: p.e., Diário da
República;
\item Integração com fontes primárias de jurisprudência: p. ex., ECLI;
\item Integração com entidades parceiras identificadas para o fornecimento de dados;
\item Interoperabilidade técnica: protocolos de comunicação, API de dados REST ou Web Service ;
\item Interoperabilidade Sintática: formato de importação e exportação de dados; deverá ser baseado
em XML e seguir normas internacionais (Akoma Ntoso XML format, uma norma OASIS para
documentos legislativos);
\item Interoperabilidade Semântica: representação semântica dos dados, ontologias OWL, utilização
do ELI e da ontologia associada;
\end{itemize}

\subsection{Identificação dos mecanismos de IA e das ferramentas conexas a usar no desenvolvimento
da plataforma}

\begin{itemize}
\item Utilização de mecanismos de Processamento de Linguagem Natural (PLN) e Mineração de
Texto, para extração de representação e significados dos textos disponíveis na base de dados
da plataforma;
\item Utilização de aprendizagem pela máquina (Machine Learning ) para incrementar a precisão do
sistema (processo de validação pelo utilizador/Configurador);
\item Definição de regras de mapeamento para aumentar a precisão das árvores de decisão
adotadas;
\item Definição de estratégia de redes neuronais para efeitos de utilização de sistemas de previsão,
designadamente na identificação de legislação e jurisprudência;
\item Identificação dos algoritmos mais adequados e o seu futuro desenvolvimento;
\item Identificação das necessidades de treino do sistema de IA.
\end{itemize}


\subsection{Requisitos da infraestrutura}

\begin{itemize}
\item Arquitetura global da plataforma;
\item Identificação dos serviços que devem compor o sistema;
\item Identificação dos requisitos técnicos de cada serviço;
\item Identificação/previsão das necessidades de processamento, espaço de armazenamento e
conectividade;
\item A Identificação de necessidade de computação em Cloud ou on-premises e respetivos
requisitos;
\item Identificação dos requisitos de interoperabilidade face a sistemas externos (comunicação,
armazenamento e representação dos dados): por exemplo, bases de dados do DRE - INCM e
de Jurisprudência dos tribunais, Ministério da Justiça (MJ), IGFEJ, 
Conselho Superior da Magistratura (CSM).
\end{itemize}


\subsection{Requisitos de sustentabilidade}



