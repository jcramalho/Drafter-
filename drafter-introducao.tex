\chapter{Introdução}

Esta atividade tem por objetivo a definição dos requisitos de uma plataforma
que auxilie na produção de atos normativos, com recurso a mecanismos de 
Inteligência Artificial (IA), em concordância
com os elementos apurados na atividade anterior do estudo 
(as melhores práticas existentes nos
sistemas de informação utilizados para apoio à redação legislativa), 
que suportem os requisitos técnicos do procedimento a lançar para a execução 
da plataforma e que deve cumprir com as caraterísticas que se descrevem nas secções seguintes.

\section{Tipologias de atos normativos}

Como já foi referido no resumo, houve necessidade de limitar as tipologias de atos normativos.
A entidade adjudicante designou como prioritárias o decreto-regulamentar, o decreto, a portaria e o despacho normativo 
(publicado na 2.a série do Diário da República) e, em certos casos, a resolução do Conselho de Ministros.

No capítulo \ref{tipologias}, apresenta-se uma análise realizada pela equipa de Direito da UM, onde se pretendeu perceber 
qual a estrutura de cada tipologia e quais os campos de metadados mais relevantes em cada uma.

\section{Requisitos}

Foram definidos, à partida, vários requisitos que se agruparam nas seguintes categorias:

\begin{itemize}
\item Requisitos funcionais de alto nível
\item Requisitos funcionais
\item Requisitos de interoperabilidade
\item Identificação dos mecanismos de IA e das ferramentas conexas a usar no desenvolvimento
da plataforma
\item Requisitos da infraestrutura
\end{itemize}

No capítulo \ref{requisitos}, faz-se uma análise detalhada de cada um.


\section{Protótipo/Prova de Conceito}

Além dos relatórios produzidos, será desenvolvida uma prova de conceito, a uma escala
reduzida, que permitirá elucidar alguns dos requisitos e, provavelmente, levantar novos requisitos ainda
não especificados.

A prova de conceito a desenvolver será composta pelas seguintes atividades e respetivos
resultados:

\begin{itemize}
\item Adoção do software open source LEOS (Legislation Editing Open Software ), como base da
solução, instalação e disponibilização online;
\item Colheita de um subconjunto de legislação do DRE;
\item Colheita de algumas bases de dados de jurisprudência dos tribunais;
\item Especificação de um modelo ontológico para a legislação colhida (baseada no trabalho já
realizado pelo EPO no ELI);
\item Processamento/Mineração, usando técnicas de NLP (Natural Language Processing ), da
legislação colhida para extração de dados para o povoamento da ontologia especificada;
\item Disponibilização da ontologia através de um motor de gestão de bases de dados orientadas a
grafos online;
\item Disponibilização de uma interface de pesquisa baseada em SPARQL que permitirá navegar na
ontologia;
\item Integração do LEOS com a base de dados ontológica: como suporte à edição de legislação;
\item  (Possibilidade) Identificar os vários tipos de documentos legislativos e estudar a hipótese de
aplicar técnicas de Machine Learning (ML) para gerar automaticamente conteúdo novo no
documento que está a ser editado.
\end{itemize}

No capítulo \ref{prototipo}, apresenta-se o trabalho realizado que conduziu à versão do protótipo em linha.
