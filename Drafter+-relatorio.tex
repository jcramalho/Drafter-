\documentclass[pdftex,12pt,a4paper]{report}
\usepackage[portuges]{babel}
\usepackage[utf8]{inputenc}
\usepackage{t1enc}
\usepackage{fancyhdr}
\usepackage{times}
\usepackage{alltt}
\usepackage{graphicx}% para poder utilizar imagens
\usepackage{fancyvrb}% para Verbatim
\usepackage{url}

\usepackage{multirow}

%-----------------------------------------------------------------------------------
%-- Definições: preencher com os valores específicos de 
%--                      cada proposta
%-----------------------------------------------------------------------------------
\newcommand{\titulo}{DRAFTER+}
\newcommand{\subtitulo}{Requisitos de uma plataforma que auxilie na produção de atos normativos}
\newcommand{\data}{24 de Abril de 2024}
\newcommand{\dataansi}{20240424}
\newcommand{\destinatario}{Direção Geral de Política da Justiça}
\newcommand{\destsigla}{DGPJ}
\newcommand{\proponente}{Universidade do Minho}
\newcommand{\keep}{\textsc{UM}}

%-------------------------------------------------------------------------
%--  Arranjo gráfico das páginas e da mancha de texto
%-------------------------------------------------------------------------
\usepackage[twoside,verbose,body={16cm,24cm},
			left=30mm,top=20mm]{geometry}
			
\pagestyle{fancy}

\fancyfoot{}% clear all defaults for footers
\fancyhead{}% clear all defaults for headers
\fancyfoot[RO]{\small \copyright \ \proponente \ \ \thepage}
\fancyfoot[LE]{\small \thepage\ \ \copyright \ \proponente}

\fancyhead[RO]{\small \leftmark} 
\fancyhead[LE]{\small \titulo}
\renewcommand{\headrulewidth}{0.3pt}% Espessura da linha que separa o header

\fancypagestyle{plain}{%
\fancyhf{} % limpa todos os headers e footers deste estilo
\fancyfoot[RO]{\small \copyright \ \proponente \ \ \thepage}
\renewcommand{\headrulewidth}{0pt}
}

%-------------------------------------------------------------------------
%--  Cabeçalhos dos capítulos
%-------------------------------------------------------------------------
\usepackage[Lenny]{fncychap}
\makeatletter 
  \renewcommand{\DOCH}{%
    \settoheight{\myhi}{\CTV\FmTi{Test}}
    \setlength{\py}{\baselineskip}
    \addtolength{\py}{\RW}
    \addtolength{\py}{\myhi}
    \setlength{\pyy}{\py}
    \addtolength{\pyy}{-1\RW}
     
    \raggedright
    \CNoV\thechapter
    \hskip 3pt\mghrulefill{\RW}\rule[-1\pyy]{2\RW}{\py}\par\nobreak}

  \renewcommand{\DOTI}[1]{%
    \addtolength{\pyy}{-4pt}
    \settoheight{\myhi}{\CTV\FmTi{#1}}
    \addtolength{\myhi}{\py}
    \addtolength{\myhi}{-1\RW}
    \vskip -1\pyy
    \rule{2\RW}{\myhi}\mghrulefill{\RW}\hskip 2pt
    \raggedleft\CTV\FmTi{#1}\par\nobreak
    \vskip 50\p@}
\makeatother

\parindent=0pt

\setlength{\parskip}{.5\baselineskip}% Distância entre paragráfos

%-------------------------------------------------------------------------
%--  Espaço entre linhas
%-------------------------------------------------------------------------
\usepackage{setspace}
\onehalfspacing
%\doublespacing

%-------------------------------------------------------------------------
%--  As minhas macros
%-------------------------------------------------------------------------
% Exemplo: \figura{fig-operacoes-rada}{./imagens/operacoes-rada.png}{CLAV: painel de operações com os RADA}{width=15cm}
\newcommand{\figura}[4]{% label, ficheiro, titulo, opções
  \begin{figure}[htp]%
  \begin{center}%
  \includegraphics[#4]{#2}%
  \end{center}%
  \textbf{\caption{\label{#1} #3}}%
  \end{figure}}

%-------------------------------------------------------------------------
%--  Citações
%-------------------------------------------------------------------------
\usepackage[font=itshape]{quoting}
%-------------------------------------------------------------------------
%--  Links internos
%-------------------------------------------------------------------------
\usepackage[colorlinks=true, linkcolor=blue, urlcolor=blue, citecolor=blue]{hyperref}
%-----------------------------------------------------------------------------------
%-- Início
 
\newcommand{\HRule}{\rule{\linewidth}{0.5mm}}
 
\begin{document}
 
\begin{titlepage}
 
\begin{tabular*}{\textwidth}%
    [b] {@{\extracolsep{0.5cm}}lr}

\begin{minipage}{0.6\textwidth}
\begin{flushleft}
% Upper part of the page
\vspace{3cm}
\includegraphics{./eeng.png}\\
\includegraphics[width=5cm]{./onto.png}\\
\end{flushleft}
\end{minipage}

&

\begin{minipage}{0.4\textwidth}

\vspace{2cm}

 \begin{flushleft}
 
\textsc{\LARGE 
\textbf{\titulo}
}\\[1.5cm]
 
\large \subtitulo
 
\vspace{6cm}

\data
 
\end{flushleft}
\end{minipage}

\\
\end{tabular*}
 
\end{titlepage}


\begin{titlepage}
 
\begin{tabular*}{\textwidth}%
    [b] {@{\extracolsep{0.5cm}}lr}

\begin{minipage}{0.6\textwidth}
\begin{flushleft}
% Upper part of the page
\vspace{3cm}
\includegraphics{./eeng.png}\\
\includegraphics[width=5cm]{./onto.png}\\
\end{flushleft}
\end{minipage}

&

\begin{minipage}{0.4\textwidth}

\vspace{2cm}

 \begin{flushleft}
 
\textsc{\LARGE 
\textbf{Relatório Técnico}
}\\[1.5cm]
 
\large Requisitos de uma plataforma que auxilie na produção de atos normativos
\vspace{3cm}
\end{flushleft}
\end{minipage}

\\
\end{tabular*}

\normalsize
\begin{center}
    \begin{tabular}{l|r}
        ID Documento      & RT-\dataansi-\destsigla \\\hline
        Versão                   & 1.0 \\\hline
        Acesso                  & Restrito \\\hline
        Data de emissão & \data \\\hline
        Autor                      & José Carlos Ramalho \\\hline
        Colaborador & Luís Filipe Cunha \\\hline
        Destinatário         & \destinatario\\\hline                         
    \end{tabular}
\end{center}

 
\end{titlepage}


\tableofcontents

\chapter{Entidade Executante}

O Departamento de Informática da Universidade do Minho (DIUM) tem por missão a divulgação do conhecimento, 
fundamental e especializado, nas áreas da ciência e das tecnologias da computação, com particular destaque para a 
Programação associada à Verificação e Segurança, os Sistemas Inteligentes, os Sistemas Distribuídos e confiáveis, 
os Sistemas de Computação de Alto-desempenho, a Engenharia de Software e as Comunicações e Redes de Computadores.

Aposta numa abordagem rigorosa à resolução de problemas por computador com base na adopção de modelos formais e 
métodos sistemáticos de análise e desenvolvimento. Cumpre a sua missão:

\begin{itemize}
    \item Lecionando cursos de licenciatura, e pós-graduação: mestrado e doutoramento;
    \item Realizando projetos de investigação e desenvolvimento internos e externos à Universidade.
\end{itemize}

Conta para isso com um pessoal permanente de cerca de 52 Docentes (todos doutorados) e 10 técnicos e mais de uma dezena 
de professores convidados para reforço das várias equipes docentes. Aos cursos que oferece, assegura um nível de ensino 
de qualidade elevada, demonstrada quer pelo avultado número de candidatos às suas ofertas formativas, quer pela grande e 
continuada procura dos estudantes formados pelo DIUM por parte dos empregadores nacionais e estrangeiros.

Para criar e manter actual o conhecimento que ensina e aplica, a actividade de investigação dos seus docentes está enquadrada 
em vários centros de investigação. Aqui exploram a teoria e desenvolvem projetos de concretização, com a colaboração de bolseiros 
de vários níveis (desde iniciação à investigação a pós-doutorados), Associação de Estudantesde pós-graduação e de pós-doutoramento.


\section{Informação de Contacto}

\begin{center}
\begin{tabular}{l|r}
Endereço Web                 & \url{http://www.di.uminho.pt} \\\hline
Telefone                            & +351 253 604430 \\\hline
Correio electrónico          & jcr@di.uminho.pt \\\hline
Responsável do projeto       & José Carlos Ramalho \\\hline
Morada                       & Departamento de Informática\\
                             & Universidade do Minho\\
                             & 4710-057 Gualtar, Braga\\\hline                         
\end{tabular}
\end{center}
\chapter{Sumário executivo}

Este documento descreve os trabalhos realizados no âmbito do levantamento de requisitos para o desenvolvimento 
de uma plataforma que deverá auxiliar na produção de atos normativos.

Este trabalho desenvolve-se no âmbito do procedimento com a Ref.ª PRR-12257-23-04 materializado num contrato 
entre a Direção Geral de Política da Justiça (DGPJ) e a Universidade do Minho (UM).

Um ato normativo é materializado num documento legislativo.
Há várias tipologias de documentos legislativos, várias dezenas.
Tratá-las todas está fora do âmbito deste projeto.
A entidade adjudicante designou como prioritárias o decreto-regulamentar, o decreto, a portaria e o despacho normativo 
(publicado na 2.a série do Diário da República) e, em certos casos, a resolução do Conselho de Ministros.

Ao longo do documento, iremos descrever o desenvolvimento do projeto identificando, para cada ponto do desenvolvimento,
os requisitos que se vão cumprindo.

\chapter{Introdução}

Esta atividade tem por objetivo a definição dos requisitos de uma plataforma
que auxilie na produção de atos normativos, com recurso a mecanismos de 
Inteligência Artificial (IA), em concordância
com os elementos apurados na atividade anterior do estudo 
(as melhores práticas existentes nos
sistemas de informação utilizados para apoio à redação legislativa), 
que suportem os requisitos técnicos do procedimento a lançar para a execução 
da plataforma e que deve cumprir com as caraterísticas que se descrevem nas secções seguintes.

\section{Tipologias de atos normativos}

Como já foi referido no resumo, houve necessidade de limitar as tipologias de atos normativos.
A entidade adjudicante designou como prioritárias o decreto-regulamentar, o decreto, a portaria e o despacho normativo 
(publicado na 2.a série do Diário da República) e, em certos casos, a resolução do Conselho de Ministros.

No capítulo \ref{tipologias}, apresenta-se uma análise realizada pela equipa de Direito da UM, onde se pretendeu perceber 
qual a estrutura de cada tipologia e quais os campos de metadados mais relevantes em cada uma.

\section{Requisitos}

Foram definidos, à partida, vários requisitos que se agruparam nas seguintes categorias:

\begin{itemize}
\item Requisitos funcionais
\item Requisitos de interoperabilidade
\item Identificação dos mecanismos de IA e das ferramentas conexas a usar no desenvolvimento
da plataforma
\item Requisitos da infraestrutura
\item Requisitos de sustentabilidade
\end{itemize}

No capítulo \ref{requisitos}, faz-se uma análise detalhada de cada um.


\section{Protótipo/Prova de Conceito}

Além dos relatórios produzidos, será desenvolvida uma prova de conceito, a uma escala
reduzida, que permitirá elucidar alguns dos requisitos e, provavelmente, levantar novos requisitos ainda
não especificados.

A prova de conceito a desenvolver será composta pelas seguintes atividades e respetivos
resultados:

\begin{itemize}
\item Adoção do software open source LEOS (Legislation Editing Open Software ), como base da
solução, instalação e disponibilização online;
\item Colheita de um subconjunto de legislação do DRE;
\item Colheita de algumas bases de dados de jurisprudência dos tribunais;
\item Especificação de um modelo ontológico para a legislação colhida (baseada no trabalho já
realizado pelo EPO no ELI);
\item Processamento/Mineração, usando técnicas de NLP (Natural Language Processing ), da
legislação colhida para extração de dados para o povoamento da ontologia especificada;
\item Disponibilização da ontologia através de um motor de gestão de bases de dados orientadas a
grafos online;
\item Disponibilização de uma interface de pesquisa baseada em SPARQL que permitirá navegar na
ontologia;
\item Integração do LEOS com a base de dados ontológica: como suporte à edição de legislação;
\item  (Possibilidade) Identificar os vários tipos de documentos legislativos e estudar a hipótese de
aplicar técnicas de Machine Learning (ML) para gerar automaticamente conteúdo novo no
documento que está a ser editado.
\end{itemize}

No capítulo \ref{prototipo}, apresenta-se o trabalho realizado que conduziu à versão do protótipo em linha.


\chapter{Tipologias de atos normativos}
\label{tipologias}

A plataforma que se pretende criar deverá ter modelos pré-criados para todas as tipologias de atos normativos 
que se venham a suportar. 
Um modelo de uma tipologia pressupõe a especificação estrutural dos documentos pertencentes a essa tipologia, 
podendo também conter uma sugestão de texto a incluir nos vários elementos estruturais que a compoem.

No contexto legístico português há dezenas de tipologias. No âmbito deste trabalho e devido a 
restrições temporais, foi necessário reduzir a um subconjunto mas significativo e representativo do que se pretende.

Consultou-se a entidade adjudicante que designou como prioritárias a lei, o decreto-regulamentar, o decreto, 
a portaria e o despacho normativo 
(publicado na 2.a série do Diário da República) e, em certos casos, a resolução do Conselho de Ministros.

\section{Metadados}

Depois de uma análise feita sobre vários documentos de cada uma destas tipologias, chegou-se ao seguinte conjunto
de metadados, comuns a todas:

\begin{description}
    \item[referencia] - fórmula usualmente utilizada para identificar os diplomas; 
    Normalmente, os documentos dentro de uma tipologia 
    são referenciados por combinação de tipologia, número de série e ano;
    \item[tipologia] - designação da tipologia: \texttt{Lei}, \texttt{Decreto-lei}, etc;
    \item[localPublicacao] - local de publicação, 1.ª ou 2.ª Série do 
    Diário da República;
    \item[numPublicacao] - número de publicação, número atribuído sequencialmente 
    dentro do mesmo ano e da mesma tipologia;
    \item[dataPublicacao] - data de publicação do diploma;
    \item[emissor] - entidade emissora; Pode ser a Assembleia da República, o Governo, 
    a Presidência do Conselho de Ministros, um Ministério em Especial ou uma Secretaria 
    de Estado de algum Ministério;
    \item[sumario] - contém a indicação do assunto principal do diploma;
    \item[preambulo] - contém um enquadramento legal e justificativo do diploma, 
    que normalmente termina com a indicação de que a entidade emissora \emph{"decreta o seguinte"};
    \item[articulado] - contém a parte dispositiva do diploma, artigos/normas legais;
    \item[anexos] - pode conter ou não; são usualmente colocadas as tabelas, listagem, mapas, símbolos, 
    ou outros elementos gráficos ou quantitativos referidos no articulado.
\end{description}

No seguimento desta análise, referem-se exemplos reais de documentos nas tipologias selecionadas e apresentam-se 
exemplos de como seriam os respetivos registos de metadados em XML. Escolhe-se o XML por ser um formato aberto,
e por ser o formato a usar na plataforma a ser desenvolvida.

\section{Lei}

Um documento desta tipologia pode ser definido como um ato legislativo, emanado pela Assembleia da República, 
no exercício da sua função legislativa, ao abrigo do artigo 164.º e 165.º da Constituição da República Portuguesa.

As leis da Assembleia da República obedecem ao formulário seguinte:
\begin{quoting}
A Assembleia da República decreta, nos termos da alínea... do artigo 161.º da Constituição, o seguinte:
(Segue-se o texto)
\end{quoting}

Tratando-se de lei constitucional ou orgânica, deve mencionar-se expressamente o termo correspondente, 
na parte final da fórmula. 

Tratando-se de resoluções de aprovação de tratados ou acordos internacionais, o texto é composto do seguinte modo:
\begin{quoting}
Aprovar (para ratificação, no caso dos tratados) o ... 
(segue-se a identificação do tratado ou do acordo internacional em forma simplificada, 
com indicação da matéria a que respeita, do local e data da assinatura, sendo o teor do respetivo 
instrumento publicado em anexo).
\end{quoting}

\subsection{Exemplo: Lei n.º 74/98, de 11 de novembro} 

O seu registo de metadados teria a seguinte extrutura em XML:

\begin{Verbatim}[frame=single, numbers=left, fontsize=\small, commandchars=\\\{\}]
<?xml version="1.0" encoding="UTF-8"?>
<documento>
    <referencia>Lei 74/98</referencia>
    <tipologia>Lei</tipologia>
    <localPublicacao>1.ª Série</localPublicacao>
    <numPublicacao>74</numPublicacao>
    <dataPublicacao>1998-11-11</dataPublicacao>
    <emissor>Assembleia da República</emissor>
    <sumario>Publicação, identificação e formulário 
            dos diplomas.</sumario>
    <preambulo>A Assembleia da República decreta, 
            nos termos da alínea c) do artigo 161.º da 
            Constituição, para valer como lei geral da República, 
            o seguinte:</preambulo>
    <articulado>ver diploma</articulado>
    <anexos>ver diploma</anexos>
</documento>
\end{Verbatim}

Este documento XML, foi criado apenas para dar uma visão minimalista dos metadados num formato inteligível quer 
para a máquina quer para o humano. Na plataforma final, estes documentos terão de estar no formato \texttt{Akoma Ntoso}.
É um formato que traz mais alguma complexidade pois prevê a interoperabilidade internacional desta informação.
A título de exemplo, apresenta-se uma possível versão do documento acima neste formato.

\begin{Verbatim}[frame=single, numbers=left, fontsize=\footnotesize, commandchars=\\\{\}]
<?xml version="1.0" encoding="UTF-8"?>
  <akomaNtoso>
    <act>
      <meta>
        <identification>
          <FRBRWork>
            <FRBRthis value="/akn/pt/act/lei/1998-11-11/74"/>
            <FRBRuri value="/akn/pt/act/lei"/>
            <FRBRdate date="1998-11-11" name="Lei 74/98"/>
            <FRBRauthor href="#assembleia-da-republica" as="author"/>
            <FRBRcountry value="pt"/>
          </FRBRWork>
          <FRBRExpression>
            <FRBRthis value="/akn/pt/act/lei/1998-11-11/74@1998-11-11"/>
            <FRBRuri value="/akn/pt/act/lei/1998-11-11/74"/>
            <FRBRdate date="1998-11-11" name="expressed"/>
            <FRBRauthor href="#assembleia-da-republica" as="author"/>
            <FRBRlanguage language="pt"/>
          </FRBRExpression>
          <FRBRManifestation>
            <FRBRthis value="/akn/pt/act/lei/1998-11-11/74@1998-11-11"/>
            <FRBRuri value="/akn/pt/act/lei/1998-11-11/74@1998-11-11"/>
            <FRBRdate date="1998-11-11" name="manifested"/>
            <FRBRauthor href="#assembleia-da-republica" as="author"/>
          </FRBRManifestation>
        </identification>
        <publication>
          <published>
            <refersTo value="1ª Série"/>
            <number value="74"/>
            <date date="1998-11-11"/>
          </published>
        </publication>
        <references source="#assembleia-da-republica">
          <TLCOrganization id="assembleia-da-republica">
            <orgName>Assembleia da República</orgName>
          </TLCOrganization>
        </references>
      </meta>
      <body>
        <preamble>
          <p>A Assembleia da República decreta, nos termos da alínea c) 
            do artigo 161.º da Constituição, para valer como lei geral da 
            República, o seguinte:</p>
        </preamble>
        <article>
          <heading>Articulado</heading>
          <p>ver diploma</p>
        </article>
        <annex>
          <heading>Anexos</heading>
          <p>ver diploma</p>
        </annex>
      </body>
    </act>
  </akomaNtoso>        
\end{Verbatim}

Explicação da estrutura:
\begin{description}
\item[akomaNtoso]: Elemento raiz que contém todo o documento;
\item[act]: Elemento que representa o ato legislativo;
\item[meta]: Metadados do documento, incluindo identificação e publicação;
\item[identification]: Identificação do documento nas três fases (Work, Expression, Manifestation);
\item[FRBRWork]: Representa a obra em si, contendo a URI e data de promulgação;
\item[FRBRExpression]: Representa a expressão da obra, incluindo a data e idioma da publicação;
\item[FRBRManifestation]: Representa a manifestação física ou digital da expressão da obra;
\item[publication]: Detalhes da publicação, como série, número e data;
\item[preface]: Contém o preâmbulo do documento;
\item[preamble]: O texto do preâmbulo;
\item[body]: Corpo principal do documento, dividido em seções;
\item[section]: Seções do documento, como sumário, articulado e anexos;
\item[heading]: Título da seção;
\item[p]: Parágrafo de texto.
\end{description}

Este exemplo cobre a estrutura básica do documento original dentro do padrão Akoma Ntoso. 
Dependendo dos detalhes específicos e dos requisitos adicionais, como normas de citação ou referências cruzadas, 
a estrutura pode ser expandida.


\section{Decreto-lei}

Um documento desta tipologia pode ser definido como um ato legislativo, diploma do Governo, no exercício 
da sua função legislativa, ao abrigo do artigo 198.º da Constituição da República Portuguesa.

Os decretos-lei têm o seguinte formulário:

\begin{description}
    \item[Decretos-leis previstos na alínea a) do n.º 1 do artigo 198.º da Constituição]: 
    
    \begin{quoting}
        Nos termos da alínea a) do n.º 1 do artigo 198.º da Constituição, o Governo decreta o seguinte:
        (Segue-se o texto.)
    \end{quoting}

    \item[Decretos-leis previstos na alínea b) do n.º 1 do artigo 198.º da Constituição]: 
    
    \begin{quoting}
        No uso da autorização legislativa concedida pelo artigo... da Lei n.º ...., de... 
        de..., e nos termos da alínea b) do n.º 1 do artigo 198.º da Constituição, 
        o Governo decreta o seguinte:
        (Segue-se o texto.)
    \end{quoting}

    \item[Decretos-leis previstos na alínea c) do n.º 1 do artigo 198.º da Constituição]: 
    
    \begin{quoting}
        No desenvolvimento do regime jurídico estabelecido pela Lei (ou Decreto-Lei) n.º ...., 
        de... de..., e nos termos da alínea c) do n.º 1 do artigo 198.º da Constituição, 
        o Governo decreta o seguinte:
        (Segue-se o texto.)
    \end{quoting}

    \item[Decretos-leis previstos no n.º 2 do artigo 198.º da Constituição]: 
    
    \begin{quoting}
        Nos termos do disposto no n.º 2 do artigo 198.º da Constituição, o Governo decreta o seguinte:
        (Segue-se o texto.)
    \end{quoting}

\end{description}

\subsection{Exemplo: DL n.º 4/2024, de 5 de janeiro} 
    
O seu registo de metadados teria a seguinte extrutura em XML:
    
\begin{Verbatim}[frame=single, numbers=left, fontsize=\small, commandchars=\\\{\}]
<?xml version="1.0" encoding="UTF-8"?>
    <documento>
        <referencia>DL 4/2024</referencia>
        <tipologia>DL</tipologia>
        <localPublicacao>1.ª Série</localPublicacao>
        <numPublicacao>4</numPublicacao>
        <dataPublicacao>2024-01-05</dataPublicacao>
        <emissor>Presidência do Conselho de Ministros</emissor>
        <sumario>Institui o mercado voluntário de carbono e 
            estabelece as regras para o seu funcionamento.</sumario>
        <preambulo>ver diploma</preambulo>
        <articulado>ver diploma</articulado>
        <anexos>ver diploma</anexos>
    </documento>
\end{Verbatim}


\section{Decreto}

Um documento desta tipologia pode ser definido como um Diploma do Governo que visa aprovar 
os acordos internacionais, ao abrigo do artigo 197.º, n.º 1, al. c), da Constituição da 
República Portuguesa.

Os decretos têm o seguinte formulário:

\begin{quoting}
    Nos termos da alínea c) do n.º 1 do artigo 197.º da Constituição, o Governo aprova 
    o... (segue-se a identificação do acordo internacional em forma simplificada, com 
    indicação da matéria a que respeita, do local e da data da assinatura, sendo o 
    teor do respetivo instrumento publicado em anexo).
\end{quoting}


\subsection{Exemplo: Decreto 1/2024, de 22 de janeiro} 
    
O seu registo de metadados teria a seguinte extrutura em XML:
    
\begin{Verbatim}[frame=single, numbers=left, fontsize=\small, commandchars=\\\{\}]
<?xml version="1.0" encoding="UTF-8"?>
    <documento>
        <referencia>Decreto 1/2024</referencia>
        <tipologia>Decreto</tipologia>
        <localPublicacao>1.ª Série</localPublicacao>
        <numPublicacao>1</numPublicacao>
        <dataPublicacao>2024-01-22</dataPublicacao>
        <emissor>Presidência do Conselho de Ministros</emissor>
        <sumario>Aprova o Acordo de Cooperação Económica entre a 
            República Portuguesa e a República da Moldova.</sumario>
        <preambulo>ver diploma</preambulo>
        <articulado>ver diploma</articulado>
        <anexos>ver diploma</anexos>
    </documento>
\end{Verbatim}


\section{Decreto-Regulamentar}

Um documento desta tipologia pode ser definido como um Regulamento. É um diploma do Governo, 
no exercício da sua função administrativa, ao abrigo dos artigos 199.º, als. c) ou g), e 
112.º, n.º 6 . É pouco comum.

Os decretos-regulamentares têm o seguinte formulário:

\begin{quoting}
    Nos termos da alínea c) do artigo 199.º da Constituição e... 
    (segue-se a identificação do ato legislativo a regulamentar), 
    o Governo decreta o seguinte: : 
    (segue-se o texto)
\end{quoting}


\subsection{Exemplo: DR 3/2024, de 21 de fevereiro} 
    
O seu registo de metadados teria a seguinte extrutura em XML:
    
\begin{Verbatim}[frame=single, numbers=left, fontsize=\small, commandchars=\\\{\}]
<?xml version="1.0" encoding="UTF-8"?>
    <documento>
        <referencia>DR 3/2024</referencia>
        <tipologia>Decreto-Regulamentar</tipologia>
        <localPublicacao>1.ª Série</localPublicacao>
        <numPublicacao>3</numPublicacao>
        <dataPublicacao>2024-02-21</dataPublicacao>
        <emissor>Presidência do Conselho de Ministros</emissor>
        <sumario>Procede à fixação do universo dos contribuintes 
            abrangidos pela declaração automática de rendimentos.
        </sumario>
        <preambulo>ver diploma</preambulo>
        <articulado>ver diploma</articulado>
        <anexos>ver diploma</anexos>
    </documento>
\end{Verbatim}


\section{Resolução do Conselho de Ministros}

Um documento desta tipologia pode ser definido como uma resolução emanada quando 
o Governo reúne em plenário, concretiza-se o Conselho de Ministros. 
É um ato normativo do Governo no exercício da sua função administrativa.

As Resoluções do Conselho de Ministros têm o seguinte formulário:

\begin{quoting}
    Nos termos da alínea... do artigo 199.º da Constituição, o Conselho de Ministros 
    resolve:
        (Segue-se o texto.)
\end{quoting}

Ou:
\begin{quoting}
    Nos termos do... (segue-se a identificação do ato e da respetiva norma que 
    estabelece a exigência de resolução) e da alínea... do artigo 199.º da Constituição, 
    o Conselho de Ministros resolve:
    (Segue-se o texto.)
\end{quoting}


\subsection{Exemplo: RCM 28/2024, de 23 de fevereiro} 
    
O seu registo de metadados teria a seguinte extrutura em XML:
    
\begin{Verbatim}[frame=single, numbers=left, fontsize=\small, commandchars=\\\{\}]
<?xml version="1.0" encoding="UTF-8"?>
    <documento>
        <referencia>RCM 28/2024</referencia>
        <tipologia>RCM</tipologia>
        <localPublicacao>1.ª Série</localPublicacao>
        <numPublicacao>28</numPublicacao>
        <dataPublicacao>2024-02-23</dataPublicacao>
        <emissor>Presidência do Conselho de Ministros</emissor>
        <preambulo>ver diploma</preambulo>
        <articulado>ver diploma</articulado>
        <anexos>ver diploma</anexos>
    </documento>
\end{Verbatim}

O campo \texttt{sumário} considera-se não aplicável nesta tipologia.


\section{Portaria}

Um documento desta tipologia pode ser definido como um diploma do Governo, 
no exercício da sua função administrativa. 
É muito comum que a própria lei determine a sua execução mediante portaria.

As Portarias têm o seguinte formulário:

\begin{quoting}
    Manda o Governo, pelo... (indicar o membro ou membros competentes), 
    o seguinte: 
    (Segue texto)
\end{quoting}


\subsection{Exemplo: Portaria 68/2024, de 23 de Fevereiro} 
    
O seu registo de metadados teria a seguinte extrutura em XML:
    
\begin{Verbatim}[frame=single, numbers=left, fontsize=\small, commandchars=\\\{\}]
<?xml version="1.0" encoding="UTF-8"?>
    <documento>
        <referencia>Portaria 68/2024</referencia>
        <tipologia>Portaria</tipologia>
        <localPublicacao>1.ª Série</localPublicacao>
        <numPublicacao>68</numPublicacao>
        <dataPublicacao>2024-02-23</dataPublicacao>
        <emissor>Presidência do Conselho de Ministros</emissor>
        <sumario>Décima segunda alteração ao Regulamento Específico 
            do Domínio da Competitividade e Internacionalização
        </sumario>
        <preambulo>ver diploma</preambulo>
        <articulado>ver diploma</articulado>
        <anexos>ver diploma</anexos>
    </documento>
\end{Verbatim}


\section{Despacho Normativo}

Um documento desta tipologia pode ser definido como um diploma do Governo, 
no exercício da sua função administrativa. 

Os Despachos Normativos têm o seguinte formulário:

\begin{quoting}
    (Inicia por identificar o acto legislativo que lhe serve de base.)
\end{quoting}


\subsection{Exemplo: DN 1/2024, de 5 de janeiro} 
    
O seu registo de metadados teria a seguinte extrutura em XML:
    
\begin{Verbatim}[frame=single, numbers=left, fontsize=\small, commandchars=\\\{\}]
<?xml version="1.0" encoding="UTF-8"?>
    <documento>
        <referencia>DN 1/2024</referencia>
        <tipologia>DN</tipologia>
        <localPublicacao>2.ª Série</localPublicacao>
        <numPublicacao>1</numPublicacao>
        <dataPublicacao>2024-01-05</dataPublicacao>
        <emissor>Gabinete do Secretário de Estado do Turismo, 
            Comércio e Serviços (Ministério da Economia e Mar)
        </emissor>
        <sumario>Prorroga o prazo de apresentação de candidaturas ao 
            concurso específico da Linha Interior + Turismo, aberto 
            na sequência dos incêndios de 4 e 5 de agosto de 2023
        </sumario>
        <preambulo>ver diploma</preambulo>
        <articulado>ver diploma</articulado>
        <anexos>ver diploma</anexos>
    </documento>
\end{Verbatim}


\section{Akoma Ntoso XML}

Akoma Ntoso (Arquitetura para Gestão Orientada ao Conhecimento de Textos Normativos Africanos usando Padrões Abertos e Ontologias) 
é uma norma técnica internacional para representar documentos executivos, legislativos e judiciais de maneira estruturada, 
utilizando um vocabulário XML específico do domínio.

O termo \texttt{akoma ntoso} significa "corações ligados" na língua Akan da África Ocidental e, por essa razão, foi escolhido 
para designar esta norma XML. A sigla usada normalmente para designar este formato é \texttt{AKN}.

\subsection{Estrutura e composição}

A norma AKN fornece uma estrutura abrangente para representar documentos parlamentares, legislativos e judiciais 
num formato XML legível quer por máquinas quer por humanos. 

Tem os seguintes componentes:

\begin{description}
\item[Vocabulário XML]: Este define o mapeamento entre a estrutura dos documentos legislativos e suas representações equivalentes 
em XML. Essencialmente, traduz os elementos dos textos legislativos num formato estruturado que pode ser facilmente processado por 
máquinas;

\item[Esquema XML]: Especificação da estrutura e respetivas restrições dos documentos legislativos em XML. 
Oferece uma capacidade descritiva extensiva aos vários tipos de documentos legislativos, incluindo:
    \begin{itemize}
        \item Documentos Parlamentares: Projetos de lei, atos e registros parlamentares.
        \item Documentos Judiciários: Sentenças e opiniões judiciais.
        \item Publicações Governamentais: Diários oficiais e outros registros governamentais.
    \end{itemize}

\item[Nomenclatura]: A AKN recomenda uma nomenclatura para identificar unicamente as fontes legislativas, 
baseada no modelo Requisitos Funcionais para Registros Bibliográficos (FRBR). 
Isso garante que cada documento possa ser identificado e referenciado de forma única, facilitando a gestão e a recuperação.

\item[Definição de Tipo MIME]: Esta especifica o tipo MIME apropriado para documentos Akoma Ntoso, garantindo que sejam corretamente reconhecidos e processados por navegadores web e outros sistemas de software.

\end{description}

Ao aderir a esses padrões, o Akoma Ntoso possibilita a criação de documentos legais interoperáveis, reutilizáveis e acessíveis, facilitando a transparência e a eficiência nos processos legais e legislativos.


\section{Sumário}

Ao longo deste capítulo, foram caraterizadas as tipologias de atos normativos que serão 
consideradas neste trabalho conducente a uma prova de conceito.

Já existe um formato aberto definido para o intercâmbio deste tipo de informação a nível europeu 
(Akoma Ntoso XML\footnote{\url{https://www.oasis-open.org/standard/akn-v1-0/}}, 
uma norma OASIS\footnote{\url{https://www.oasis-open.org/org/}} para documentos legislativos).
Esse formato está especificado num XML Schema, por isso se adiantou com uma possível
instância XML para cada um dos tipos de registo.



\chapter{Requisitos}
\label{requisitos}

Na proposta subjacente ao contrato ao abrigo do qual se redige este relatório, constavam vários grupos de requisitos.
Neste capítulo, contextualiza-se cada um deles de acordo com o protótipo (cap. \ref{prototipo}) instalado, parametrizado 
e desenvolvido.

Relativamente ao protótipo, era um requisito que este se baseasse na plataforma LEOS (\emph{"Legislation Editing Open Software"}).
O LEOS tem já desenvolvidos e implementados muitos dos requisitos, carecendo de uma adaptação à realidade portuguesa.
Ao longo deste capítulo, faz-se uma associação de cada um dos requisitos a uma ou mais funcionalidades da plataforma LEOS,
indicando as parametrizações que será necessário fazer. Para os casos em que a associação não seja possível indicam-se as 
funcionalidades a desenvolver.

A criação desta nova plataforma pressupõe também uma ligação automática (interoperabilidade) a outro sistema onde estaria 
a jurisprudência nacional. Para isso, seria necessário que existisse uma normalização sobre a jurisprudência e que esta fosse 
disponibilizada numa API de dados. Algo que poderá acontecer num futuro próximo.

\section{Requisitos}

Nas subsecções seguintes descrevem-se os conceitos associados a cada requisito, se estão ou não presentes no LEOS e,
o que será preciso fazer caos não estejam.

\subsection{Requisitos funcionais}

Alguns destes requisitos estão relacionados com políticas sobre a adoção de normas e a especificação e 
criação de processos, no entanto,
ao colocarmos o LEOS como base, algumas destas decisões já foram tomadas e parte deste trabalho já está feita.

\subsubsection{Deteção automática de cumprimento de normas de produção de atos normativos}

O LEOS precisa de ser parametrizado com os modelos estruturais dos atos normativos que se pretendem criar.
No capítulo anterior, apresentou-se uma análise prévia necessária à criação destes modelos. 
O passo seguinte, será 
especificar esses modelos em XML numa linguagem específica desenvolvida para o efeito \cite{VPCB2018}.
Apresentam-se exemplos no capítulo referente à configuração da plataforma \ref{modelos}.

Esta especificação de cada um dos modelos pode e deve incluir os requisitos estruturais que devem ser observados na 
produção de atos normativos. Ao fazê-lo, o LEOS ou outra plataforma que os venha a utilizar pode verificar automaticamente 
o cumprimento dos requisitos/normas.

A linguagem definida pela comunidade para ser usada neste tipo de plataformas, o AKN (\ref{akn}), permite 
ir além dos requisitos estruturais podendo ser especificados requisitos de semântica dinâmica (dependentes 
do conteúdo que se vai introduzindo no documento).

\subsubsection{Observação de interpretações firmadas em jurisprudência sobre normas}

Este requisito pressupõe a verificação de algumas condições estruturais e semânticas.
As condições estruturais podem e devem estar refletidas no modelo criado em AKN (\ref{akn}) para a tipologia.
As condições semânticas podem ser divididas em duas: estáticas e dinâmicas.
Ambas terão de ser especificadas no momento da definição do modelo para a tipologia. 
A semântica estática pode ficar definida no modelo em AKN ou ainda numa estensão a este modelo criada na plataforma LEOS.
A semântica dinâmica, depende de valores que serão introduzidos aquando da criação do documento e a sua verificação terá 
de ser garantida usando uma parte da linguagem AKN, ou uma linguagem extra de regras ou materializadas 
no código da aplicação. Estas três alternativas dão solução ao problema, caberá à equipa de projeto 
selecionar a metodologia que for mais conveniente.

\subsubsection{Apoio à elaboração de tarefas de avaliação normativa}

As tarefas de avaliação normativa pretendem avaliar o impacto da nova legislação que está a ser criada.
É algo externo ao ato normativo e para o qual é necessário a interpretação e análise humana.
O apoio referido será no registo da informação que possa conduzir a um valor final de impacto. 

O LEOS não possui qualquer suporte à elaboração destas tarefas.
Para isso, será necessário desenvolver um componente de raiz, que poderá estar integrado no LEOS ou comunicar com este 
sendo um serviço externo.

O desenvolvimento deste componente carece de uma especificação cuidada de requisitos.

\subsubsection{Automatização dos processos de avaliação legislativa para textos preparados na plataforma}

Relacinonado com o ponto anterior. Estas funcionalidades deverão fazer parte do novo componente a desenvolver.

\subsubsection{Verificação e validação das referências normativas e legais identificadas nos textos preparados
na plataforma (verificação da existência das normas invocadas)}
\label{bases_legis}

Como já foi referido em cima, para o cumprimento deste requisito é necessário dispôr de uma base de dados com a 
produção jurídico-normativa de Portugal.

Para isso, é necessário garantir a adoção de várias regras/normas:

\begin{itemize}
    \item É preciso garantir uma forma única de referenciar os atos normativos, seja ela o ELI 
    ("\emph{European Legislation Identifier}") \cite{ELI}, ou simplesmente aquilo a que estamos habituados em Portugal, ou 
    seja, uma referência composta por tipologia, número de série e ano (no caso desta última, deverá 
    estar formalmente definida sem margem para ambiguidades); 

    \item Garantir a presença \emph{online}, de um repositório de atos normativos permanentemente atualizado e acessível;
    \item Ter uma API ("\emph{Application Program Interface}") bem definida sobre o repositório anterior que permita a 
    integração automática e a interoperabilidade com outros sistemas, por exemplo, o que se pretende desenvolver.
\end{itemize}

Devido à inexistência deste contexto de normalização, podemos avançar com um pequeno protótipo próprio 
mas que terá sempre as desvantagens de 
estar fora dos canais oficiais e de sofrer de permanente desatualização.


\subsubsection{Pesquisa e identificação automática de legislação e jurisprudência}
\label{bases_jurisprud}

Os requisitos descritos no ponto anterior são também válidos neste ponto.
Havendo um repositório de legislação que obedeça a um conjunto de regras transversais, basta incluir uma funcionalidade de 
pesquisa na sua API.

Relativamente à jurisprudência, a situação é semelhante à da legislação, é necessário garantir a adoção de várias regras/normas:

\begin{itemize}
    \item É preciso garantir uma forma única de referenciar os processos. Para isto já existe o ECLI 
    ("\emph{European Case Law Identifier}") \cite{ECLI}. É preciso promover/decretar a sua adoção à escala nacional; 

    \item Garantir uma normalização a nível dos metadados à escala nacional, todas as instituições produtoras de 
   jurisprudência deverão produzi-la na mesma forma, o que não se passa atualmente;

    \item Garantir a presença \emph{online}, de um repositório de jurisprudência permanentemente atualizado e acessível;

    \item Ter uma API ("\emph{Application Program Interface}") bem definida sobre o repositório anterior que permita a 
    integração automática e a interoperabilidade com outros sistemas, por exemplo, o que se pretende desenvolver.
\end{itemize}

À semelhança do ponto anterior, devido à inexistência deste contexto normalizado, podemos avançar 
com um pequeno protótipo próprio mas que terá sempre as desvantagens de 
estar fora dos canais oficiais e de sofrer de permanente desatualização.

\subsubsection{Verificação semântica das normas invocadas}

Para suportar este requisito será preciso ir mais além da criação dos modelos AKN para as tipologias.
Será necessário criar um modelo semântico para cada uma, aquilo que se designa por ontologia.

Neste contexto, uma ontologia pode definir-se como uma especificação formal de conhecimento de um determinado domínio bem 
caraterizado e que inteligível quer para máquinas quer para humanos.

Há várias linguagens que permitem a criação de ontologias com níveis diferentes de aplicabilidade.
Aquela que melhor se adapta à complexidade do contexto jurídico-normativo é a "\emph{Ontology Web Language} (OWL)" 
\cite{owl,owl2}.

A OWL é uma linguagem de anotação desenhada para especificar semântica, publicar e partilhar dados. 
Foi desenvolvida pelo \emph{World Wide Web Consortium} (W3C).

Um ontologia OWL é um conjunto de conceitos, seus atributos e relações entre eles. 
Adicionalmente, permite a especificação de regras semânticas via axiomas.

As ontologias são uma maneira de modelar o conhecimento de uma forma estruturada, permitindo que diferentes sistemas 
compreendam e usem esses dados de forma interoperável.

A linguagem OWL é baseada em \emph{Description Logic} (DL), que fornece uma base formal para o raciocínio automatizado. 
Isso significa que a partir de uma ontologia de base a máquina consegue inferir novos fatos a partir dos dados existentes, 
aumentando automaticamente o seu conhecimento e com isso 
melhorar a busca de informação e facilitar a integração de dados de diversas fontes.
A linguagem tem vários níveis de complexidade deixando a quem a utiliza a decisão de qual o nível que pretende usar. 
Desta forma, pode-se dizer que tudo se pode especificar em OWL.

Esta tarefa carece de uma especificação formal dos modelos semânticos que deve ser feita em conjunto com a 
especificação formal dos modelos das tipologias.
O modelo semântico OWL poderá até, inicialmente, ser povoado com a informação proveniente da especificação 
da tipologia.

\subsubsection{Descrição dos workflows para a criação e gestão de atos normativos}

O Ministério da Justiça possui já uma ferramenta própria para automatização de workflows. 
Se for possível colocar o novo sistema a dialogar com essa ferramenta consegue-se cumprir este requisito e 
poupar imenso tempo de desenvolvimento.

Para isso, será necessário garantir a interoperabilidade técnica e sintática entre os dois sistemas. 
Ou seja, terá de haver mecanismos/protocolos para o intercâmbio de informação entre os dois sistemas e 
deverá estar definido o formato da informação que se vai trocar.


\subsection{Requisitos de interoperabilidade}

Hoje em dia, falar de interoperabilidade é quase um requisito mas muitos desconhecem que esta é complexa 
e pode ser considerada em vários e diferentes níveis.

Da bibliografia, interoperabilidade é a capacidade de um sistema (informatizado ou não) de comunicar de 
forma transparente (ou o mais próximo disso) com outro sistema (semelhante ou não).

A nossa Agência para a Modernização Administrativa (AMA)  definiu um modelo para a interoperabilidade na 
Administração Pública (AP):
\begin{itemize}
\item Neste contexto, a interoperabilidade consiste na capacidade das organizações interagirem 
e agirem em prol de benefícios comuns, através de comunicação e partilha de informação e conhecimento;
\item O modelo foi baseado na \emph{Framework Europeia de Interoperabilidade} \cite{EIF}, criada pela Comissão Europeia;
\item Está organizado em quatro camadas: legal, organizacional, técnica e semântica.
\end{itemize}

Apesar de independentes, as quatro camadas são interdependentes, sendo que para se conseguir atingir a interoperabilidade 
semântica, as outras terão de estar contempladas (neste projeto é importante atingir o patamar semântico).

No contexto da interoperabilidade, pretende-se uma integração com as fontes primárias fundamentais, 
designadamente bases de dados, com toda a
legislação e atos normativos, para apoio à redação legislativa e normativa: p.e., Diário da
República. Como já foi referido atrás (\ref{bases_legis}), este é um requisito com implicações
organizacionais e políticas, é preciso normalizar e adoptar transversalmente algumas regras. Antes da 
interoperabilidate técnica, sintática e semântica, temos de resolver os requisitos da 
interoperabilidade organizacional.

Pretende-se também a integração com fontes primárias de jurisprudência, que é também um problema 
que começa no topo com a interoperabilidade organizacional (\ref{bases_jurisprud}). É preciso normalizar 
e adotar transversalmente um conjunto de regras já expostas anteriormente.

A interoperabilidade sintática e semântica fica garantida com a adoção do Akoma Ntoso \cite{AkomaNtoso2018} 
quer para a definição estrutural e semênticas das tipologias legísticas como para o seu conteúdo.

Por fim, para garantir a interoperabilidade técnica será preciso expor uma API de dados REST ou Web
Service a partir do LEOS, o que com algum trabalho se consegue fazer.


\subsection{Identificação dos mecanismos de IA e das ferramentas conexas a usar no desenvolvimento
da plataforma}

Relativamente a este assunto, a equipa do LEOS, esta equipa faz parte do grupo técnico que está a 
desenvolver soluções de interoperabilidade na Comunidade Europeia, realizou algum trabalho muito 
importante e que permite avançar muito na compreensão e do que importa desenvolver nesta área.

Durante a segunda metade de 2023, reuniram um grupo de trabalho composto por advogados, 
políticos ligados à criação de novas políticas e linguístas, com o objetivo de identificar 
as funcionalidades inteligentes que eles considerassem mais úteis no contexto em causa.
O seu contributo permitiu a identificação das seguintes funcionalidades:

\begin{itemize}
    \item Correlação entre considerandos e os termos do ato normativo em construção;
    \item Identificar automaticamente a legislação existente relevante para o ato em desenvolvimento;
    \item Identificar siglas, organizações e outras abreviaturas;
    \item Usar formulações linguísticas corretas dentro da estrutura do documento;
    \item Formulação correta de acordo com o Guia de Estilo em Português (ou outro que venha a ser produzido);
    \item Detectar divergências entre diferentes traduções linguísticas;
    \item Sugerir formulações linguísticas em disposições;
    \item Detectar e evitar estruturas que possam criar problemas na interpretação jurídica;
    \item Correlação entre atos anteriores e o novo que se está a criar;
    \item Detectar obrigações, direitos, permissões e  penalidades;
    \item Geração de texto jurídico com base em LLM (\emph{"Large Language Model"}: modelo de linguagem de grande escala).
\end{itemize}


\subsection{Requisitos da infraestrutura}

\begin{itemize}
\item Arquitetura global da plataforma;
\item Identificação dos serviços que devem compor o sistema;
\item Identificação dos requisitos técnicos de cada serviço;
\item Identificação/previsão das necessidades de processamento, espaço de armazenamento e
conectividade;
\item A Identificação de necessidade de computação em Cloud ou on-premises e respetivos
requisitos;
\item Identificação dos requisitos de interoperabilidade face a sistemas externos (comunicação,
armazenamento e representação dos dados): por exemplo, bases de dados do DRE - INCM e
de Jurisprudência dos tribunais, Ministério da Justiça (MJ), IGFEJ, 
Conselho Superior da Magistratura (CSM).
\end{itemize}


\subsection{Requisitos de sustentabilidade}





\chapter{LEOS: Legislation Editing Open Software}

\section{Introdução}

O LEOS é um projeto no âmbito da iniciativa \emph{"Interoperable Europe"} da Comissão Europeia para uma política reforçada de 
interoperabilidade do setor público, financiado pelo Programa Digital \emph{Europe (DIGITAL)} e criado para atender à 
necessidade da administração pública e das Instituições Europeias de gerar projetos de legislação em formato XML jurídico.

O projeto LEOS concentra-se em apoiar o co-desenvolvimento, co-design e co-implementação de um "ecossistema de Tecnologias de 
Informação (TI) centrado num LEOS aumentado".

O LEOS foi criado para abordar a modernização e transformação digital da elaboração e revisão de legislação nas 
Instituições da UE, agências e órgãos da UE e Estados-Membros.

Esta plataforma garante que o conteúdo elaborado pelos utilizadores siga as diretrizes de redação, oferecendo recursos como a
aplicação de estruturas de documento pré-definidas, layout pré-definido e regras de numeração. 
Tudo isso para garantir que o autor se possa focar na elaboração do texto e muito menos na gestão do layout (ou verificação). 
Para facilitar a colaboração online eficiente, o LEOS também possui outros recursos como comentários, sugestões, 
controle de versão, edição colaborativa, etc.



%\input{api}

%\input{autenticacao}

%\input{interface-RADA}

%\input{export}

%\input{conclusao}

\bibliographystyle{alpha}
\bibliography{drafter}

\end{document}